\documentclass[a5paper]{article}
% Uncomment the following line to allow the usage of graphics (.png, .jpg)
%\usepackage[pdftex]{graphicx}
% Comment the following line to NOT allow the usage of umlauts

\pagestyle{empty}
\usepackage[T2A]{fontenc}
\usepackage[utf8]{inputenc}
\usepackage[russian]{babel}
\usepackage{cmap}
\usepackage{amsthm}
\usepackage{amsmath}
\usepackage{units}
\usepackage{fancyhdr}
\usepackage{forloop}
\usepackage{amssymb}
\usepackage{url}
\usepackage{hyperref}
\usepackage{xcolor}
\usepackage[inline]{enumitem}
\usepackage{graphicx}
\usepackage{caption}
\usepackage{subcaption}
\usepackage{amscd}
\usepackage[ruled,vlined]{algorithm2e}

%------------- TIKZ --------------------
\definecolor{bblue}{rgb}{0.2, 0.4, 0.8}
\definecolor{bgreen}{rgb}{0.2, 0.6, 0.4}
\definecolor{bred}{rgb}{0.8, 0.4, 0.2}
\definecolor{bviolet}{rgb}{0.7, 0.2, 0.7}
\definecolor{blackred}{rgb}{0.6, 0.3, 0.3}
\definecolor{blackblue}{rgb}{0.3, 0.3, 0.6}

\usepackage{tikz}
\usetikzlibrary{fit,arrows,trees,shapes,snakes,shapes.geometric,calc}
\tikzset{
  treenode/.style = {align=center, inner sep=0pt, text centered,
    font=\sffamily},
  arn_nn/.style = {treenode, circle, bblue, draw=bblue, 
    fill=bblue!10,
    minimum width=0.5em, minimum height=0.5em
},% arbre rouge noir, noeud rouge
  arn_n/.style = {treenode, circle, bblue, draw=bblue, 
    text width=1.5em, very thick,
    fill=bblue!10},% arbre rouge noir, noeud rouge
  arn_g/.style = {treenode, circle, bgreen, draw=bgreen, 
    text width=1.5em, very thick,
    fill=bblue!10},% arbre rouge noir, noeud rouge
  arn_r/.style = {treenode, circle, bred, draw=bred, 
    text width=1.5em, very thick,
    fill=bred!10},% arbre rouge noir, noeud rouge
  arn_x/.style = {treenode, triangle, draw=black,
    minimum width=0.5em, minimum height=0.5em},% arbre rouge noir, nil
  triangle/.style = {treenode, bred, draw=bred, fill=bred!20, regular polygon, regular polygon
    sides=3, very thick, text width=1.5em }
}

\renewcommand{\thesection}{\arabic{section}}

\renewcommand{\headrulewidth}{0.4pt}
\renewcommand{\footrulewidth}{0.4pt}

\fancyhead[R]{\thepage}
%For multipage documents only!
%\fancyfoot[L]{page: \thepage}
\fancyhead[L]{Перечислительная комбинаторика}
\fancyfoot[C]{\topic}
\pagestyle{fancy}

\renewcommand{\baselinestretch}{1.0}
\renewcommand\normalsize{\sloppypar}

\setlength{\topmargin}{-0.5in}
\setlength{\textheight}{6.5in}
\setlength{\oddsidemargin}{-0.3in}
\setlength{\evensidemargin}{-0.3in}
\setlength{\textwidth}{4.5in}
\setlength{\parindent}{0ex}

\newcounter{problemset}
\newcounter{totalpages}
%Here you should set the total number of pages
\setcounter{totalpages}{1}

\def \topic {Экзамен по курсу перечислительной комбинаторики.
Тестовый вариант.}

\def \Z {\mathbb Z}
\def \R {\mathbb R}
\def \P {\mathbb P}
\def \C {\mathbb C}
\def \vec {\boldsymbol}

\theoremstyle{definition}
\newtheorem{lemma}{Лемма}
\newtheorem{example}{Пример}
\newtheorem{corollary}{Следствие}
\newtheorem{problem}{Задача}
\newtheorem*{theorem}{Теорема}
\newtheorem*{definition}{Определение}
\newtheorem*{solution}{Решение}

\usepackage{titlesec}

\makeatletter
\renewcommand{\section}{\@startsection
{section}%                   % the name
{1}%                         % the level
{\z@}%                       % the indent / 0mm
{-\baselineskip}%            % the before skip / -3.5ex \@plus -1ex \@minus 
%%-.2ex
{0.5\baselineskip}%          % the after skip / 2.3ex \@plus .2ex
{\centering\large\scshape}} % the style

\renewcommand{\subsection}{\@startsection
{subsection}%                % the name
{1}%                         % the level
{\z@}%                       % the indent / 0mm
{-\baselineskip}%            % the before skip / -3.5ex \@plus -1ex \@minus 
%%-.2ex
{0.5\baselineskip}%          % the after skip / 2.3ex \@plus .2ex
{\centering\large\scshape}} % the style
\makeatother

\def\seq{\text{\textsc{seq}}}   
\def\cyc{\text{\textsc{cyc}}}   
\def\set{\text{\textsc{set}}}   
\def\C{\mathbb C}

\begin{document}

\begin{center}

\newcommand{\HRule}{\rule{\linewidth}{0.5mm}}
\HRule \\[0.2cm]
{ \Large \bfseries \topic} %\\[0.2cm]
\HRule

\end{center}

\begin{minipage}{0.6\textwidth}
\begin{problem}
Рассмотрим множество \emph{плоских корневых деревьев}, содержащих \( n \) вершин. В этой задаче 
под плоским корневым деревом понимается дерево (граф без циклов), у которого
выделена одна вершина (мы называем её корневой вершиной), и граф ``нарисован на
плоскости'', то есть порядок смежных вершин по часовой стрелке имеет значение.
\end{problem}

\end{minipage}%
\hfill%
\begin{minipage}{0.4\textwidth}
\raggedleft
\begin{tikzpicture}[>=stealth',level/.style={sibling distance = 1.5cm/#1,
  level distance = 1.5cm, thick}] 
\draw

node[triangle]{\( T \)}
++(1.1,0)
node{\( \boldsymbol = \)}
++(1,0.5)
node(up)[arn_n]{\( \bullet \)}
    child{
        node[triangle]{\( T \)}
    }
    child{
        node[triangle]{\( T \)}
    }
++(0,-1.5)
node{\( \cdots \)}
++(0,-0.65)
node{\(\underbrace{\phantom{\qquad \qquad \qquad \qquad}}_{\geq 0} \)}
;
\end{tikzpicture}
\end{minipage}
\vspace{.05cm}

Выберем случайное дерево случайно равновероятно среди множества всех таких
деревьев на \( n > 1 \) вершинах. Докажите, что математическое ожидание количества
\emph{листьев} в таком дереве в точности равно \( \frac{n}{8} \).
\begin{solution}
Будем считать, что в каждом дереве содержится хотя бы одна вершина (это не
влияет на ответ задачи при \( n > 1 \), но упростит нам вычисления).
Обыкновенная производящая функция \( T(x) \) для деревьев (Каталана)
удовлетворяет уравнению
\begin{equation}
    T(x) = x \cdot \dfrac{1}{1- T(x)}
    \enspace .
\end{equation}
Чтобы ответить на вопрос задачи, недостаточно просто решить это уравнение, но мы
на всякий случай это сделаем. Я хочу обратить внимание на то, что пристальное
вглядывание в это уравнение позволяет нам убедиться, что оно имеет единственное
решение в классе формальных степенных рядов. Действительно, мы видим, что \( T(0) =
0 \), то есть свободный член равен нулю. Поэтому разложение \( \dfrac{1}{1 -
  T(x)} \) начинается с единицы, и коэффициент при \( x \) в правой части равен
\( 1 \). Имея эту информацию, можно подставить \( T(x) = x + Q(x) \), получая
\begin{equation}
   T(x) = \dfrac{x}{1 - T(x)} = \dfrac{x}{1 - x - Q(x)} = 
    x (1 + (x + Q(x)) + \ldots) = x + x^2 + \ldots
\end{equation}
Таким образом, за второй шаг подстановки мы можем понять, что коэффициент при \(
x^2 \) равен \( 1 \). 

Уравнение является квадратным относительно \( T(x) \).
Если домножить обе части на знаменатель, то мы получаем
\begin{equation}
    T(x) - T(x)^2 = x
\end{equation}
Это уравнение имеет два решения, но у одного из решений, если разложить его в
формальный степенной ряд, появляются отрицательные коэффициенты, поэтому оно не
является производящей функцией количества таких деревьев.
Выписываем то решение, которое является  нужным нам формальным
степенным рядом
\begin{equation}
    T(x) = \dfrac{1 - \sqrt{1 - 4x}}{2} = -\sum_{n \geq 1} \dfrac{(-4)^{n}}{2} {1/2
\choose n} x^n \enspace .
\end{equation}
\def\uu{ {\color{blackred}u} }
Затем рассмотрим производящую функцию от двух переменных \( T(z, \uu) \), 
где переменная \( \uu \) маркирует листья дерева, то есть коэффицент
\( [z^n \uu^k] T(z, \uu) \) равен количеству деревьев на \( n \) вершинах, у
которых имеется \( k \) листьев. Как построить такую функцию? Нужно лишь учесть
листовые вершины в конструкции дерева.

\begin{figure}[hbt]
\centering
\begin{tikzpicture}[>=stealth',level/.style={sibling distance = 1.5cm/#1,
  level distance = 1.5cm, thick}] 
\draw
node[triangle]{\( T \)}
++(1.1,0.2)
node{\( \boldsymbol = \)}
++(1,0)
node[arn_g](marked){ \( \uu \) }
++(1,0)
node{\( \boldsymbol + \)}
++(1,0.5)
node(up)[arn_n]{\( \bullet \)}
    child{
        node[triangle]{\( T \)}
    }
    child{
        node[triangle]{\( T \)}
    }
++(0,-1.5)
node{\( \cdots \)}
++(0,-0.65)
node{\(\underbrace{\phantom{\qquad \qquad \qquad \qquad}}_{\geq 1} \)}
;
\node[rectangle,dashed,draw,fit=(marked),
rounded corners=3mm,inner sep=8pt, bgreen] {};
\end{tikzpicture}
\end{figure}
Получаем уравнение на производящую функцию
\begin{equation}
    T(z, \uu) = z \uu + \dfrac{z T(z, \uu)}{1 - T(z, \uu)}
    \enspace ,
\end{equation}
и это уравнение тоже квадратное, и снова легко решается:
\begin{equation}
    T(z, \uu) = \dfrac{z\uu - z + 1 - \sqrt{(z\uu-z+1)^2-4z\uu}}{2}
    \enspace .
\end{equation}
Можно заметить, что подстановка \( \uu = 1 \) возвращает нас к непомеченным
деревьям, значит, мы всё делаем правильно.

При этом та самая производящая функция моментов, которая возникает в теории
вероятности, здесь поможет нам посчитать матожидание. В действительности, 
если обозначить \( L_n \) случайную величину, которая равна количеству листьев в
случайно выбранном дереве размера \( n \), то для неё выполнено
\begin{equation}
    \mathbb E u^{L_n} = \dfrac{[z^n] T(z, \uu)}{[z^n]T(z,1)} 
    \enspace ,
\end{equation}
Значит, матожидание можно найти, продифференцировав обе части:
\begin{equation}
    \mathbb E L_n = \left. \dfrac{d}{d \uu} \mathbb E u^{L_n} \right|_{\uu = 1}
    = \dfrac
        {\left. [z^n] \dfrac{d}{d \uu} T(z, \uu)\right|_{\uu = 1}}
        {[z^n] T(z, 1)}
    \enspace .
\end{equation}
Дифференцируем числитель и подставляем \( \uu = 1 \). Получаем:
\begin{equation}
    T'_{\uu}(z, 1) = z + \dfrac{z}{2 \sqrt{1 - 4z}} \enspace .
\end{equation}
Отношение числителя и знаменателя можно выразить с помощью биномиальной теоремы:
\begin{multline*}
    \dfrac
        { [z^n] \tfrac12 z(1-4z)^{-1/2} }
        { [z^n] -\tfrac12 (1-4z)^{1/2} }
    = 
    \dfrac
        { (-4)^{n-1} {-1/2 \choose n-1} }
        { -(-4)^n {1/2 \choose n} }
    \\=
    \dfrac{n! (-\tfrac12)(-\tfrac12-1) \ldots (-\tfrac12 - (n-1) + 1)}
    {4(n-1)! (\tfrac12)(\tfrac12 - 1) \ldots (\tfrac12 - n + 1)}
    = \dfrac{n}{8}
    \enspace .
\end{multline*}
Таким образом, при \( n \geq 2 \) математическое ожидание доли листьев в случайно
выбранном дереве составляет в точности \( \dfrac18 \).

\end{solution}

\noindent\begin{minipage}{0.6\textwidth}
\begin{problem}
\emph{Возрастающий алмаз} это
направленный ациклический граф (то есть граф, в котором рёбра имеют ориентацию, и
не существует цикла, идущего по рёбрам графа),
построенный на вершинах с различными метками от \( 1 \) до \( n \), имеющий
\textit{исток} (вершина, из которой можно попасть куда угодно) и \textit{сток}
(вершина, в которую можно попасть откуда угодно), причём граф обладает
свойством, что вдоль каждого пути номера меток возрастают.
\begin{enumerate}
\item Докажите, что экспоненциальная производящая функция количества
возрастающих алмазов имеет вид
\begin{equation}
    f(z) = \log \dfrac{1}{1 - \sin z}
    \enspace .
\end{equation}
\end{enumerate}
\end{problem}%
\end{minipage}%
\hfill%
\begin{minipage}{0.3\textwidth}\raggedleft
\begin{tikzpicture}[>=stealth',thick, node distance=1.5cm] 
   \draw           
   node[arn_r](1)                           { \( 1 \) }
   node[arn_n](2)  [below right of = 1    ] { \( 2 \) } 
   node[arn_n](4)  [below left of  = 1    ] { \( 4 \) } 
   node[arn_n](5)  [below left of  = 2    ] { \( 5 \) } 
   node[arn_n](3)  [below right of = 2    ] { \( 3 \) } 
   node[arn_n](7)  [below of       = 5    ] { \( 7 \) } 
   node[arn_n](8)  [below right of = 7    ] { \( 8 \) } 
   node[arn_n](6)  [below left  of = 7    ] { \( 6 \) } 
   node[arn_r](9)  [below left  of = 8    ] { \( 9 \) } 
   ;
   \path[->] (1) edge [bblue, thick] node {} (2);
   \path[->] (1) edge [bblue, thick] node {} (4);
   \path[->] (2) edge [bblue, thick] node {} (5);
   \path[->] (2) edge [bblue, thick] node {} (3);
   \path[->] (5) edge [bblue, thick] node {} (7);
   \path[->] (3) edge [bblue, thick] node {} (8);
   \path[->] (7) edge [bblue, thick] node {} (8);
   \path[->] (4) edge [bblue, thick] node {} (6);
   \path[->] (6) edge [bblue, thick] node {} (9);
   \path[->] (8) edge [bblue, thick] node {} (9);
\end{tikzpicture}
\end{minipage}
\begin{enumerate}
\item[2.] 
Докажите, что при больших \( n \)
количество возрастающих алмазов \( B_n \), построенных на \( n \) вершинах,
удовлетворяет свойству
\begin{equation}
    B_n \underset{n \to \infty}{\sim} 2 n! \left( \frac{2}{\pi} \right)^n
    \enspace .
\end{equation}
\end{enumerate}
\begin{solution}

\end{solution}

\begin{problem}
Рассмотрим множество всех путей на плоскости из точки \( (0,0) \) в точку \(
(n,n) \), которые состоят только из отрезков вида ``вправо'' \( (1,0) \) и ``вверх'' \(
(0,1) \). Известно, что число таких путей равно \( {2n \choose n} \), как нас
учили в курсе комбинаторики. Выберем среди всех таких путей из \( (0,0) \) в \(
(n,n) \) один путь случайно равновероятно. Количество пересечений этого пути с
диагональю (то есть количество точек вида \( (i,i) \), которые лежат на этом
пути), это случайная величина, которую мы обозначим \( U_n \). Докажите, что
матожидание \( U_n \) имеет асимптотику
\begin{equation}
    \mathbb E U_n \sim \sqrt{\pi n}
\end{equation}
\end{problem}

\begin{solution}
Задача с mathoverflow. Найти количество пересечений (возможно касаний) с
диагональю путь в квадрате \( n \times n \).               

Consider generating function of lattice paths which don't touch and don't cross the diagonal:
$$
    Cat(z) = \dfrac{1 - \sqrt{1 - 4z}}{2} = z + z^2 + 2z^3 + 5z^4 + 14z^5 + \ldots
$$
Then we take product of $ k $ Catalan generating series which stands for lattice paths touching or crossing exactly $ k$ points:
$$
    [z^n] (Cat(z))^k = \text{\# lattice paths from $(0,0)$ to $(n,n)$ having $k$ diagonal points}
$$
Then we assemble everything into bivariate generating function, adding multiple $2^k$ for two different possible sides of Catalan paths:
$$
    F(z,u) = \sum_{k \geq 1} 2^k(Cat(z))^k u^k = \dfrac{1}{1 - u(1 - \sqrt{1-4z})}
$$
Note that if we plug $u=1$ then we obtain all possible paths from $(0,0)$ to $(n,n)$ regardless of number of diagonal crosses:
$$
    \dfrac{1}{\sqrt{1-4z}} = 1 + 2z + 6z^2 + 20z^3 + \ldots = \sum_{n \geq 0} {2n \choose n} z^n \enspace .
$$
According to the method of bivariate generating functions, the expectation is expressed by applying the point derivative operator:
$$
    u \dfrac{d}{du} \sum_{n, k, \geq 0} a_{n,k} z^n u^k = 
    \sum_{n, k, \geq 0} k a_{n,k} z^n u^k \enspace ,
$$
Denoting by $[z^n]f(z)$ the coefficient at $z^n$ in function $f(z)$, and by $U_n$ – random variable denoting the number of crossings, we obtain
$$
    \mathbb E U_n = \dfrac{[z^n] \left.u \dfrac{d}{du} F(z,u)\right|_{u=1}}{[z^n] \left.F(z,u)\right|_{u=1}} \enspace .
$$
$$
    \left.\dfrac{d}{du}\dfrac{1}{1 - u(1 - \sqrt{1-4z})} \right|_{u=1} = 
    \left.\dfrac{1 - \sqrt{1-4z}}{\left[1 - u(1 - \sqrt{1-4z})\right]^2} \right|_{u=1} =
    \dfrac{1 - \sqrt{1-4z}}{1 - 4z}
$$
Applying asymptotic extraction of coefficients,
$$
    \mathbb E U_n =
    \dfrac{[z^n] \dfrac{1}{1 - 4z}}
    {[z^n] \dfrac{1}{\sqrt{1 - 4z}}} = 
    \dfrac{4^n}{ {2n \choose n} } \sim \sqrt{ \pi n}
$$
\end{solution}
    
\begin{problem} Рассмотрим множество бинарных строк \( \{ a, b \}^\ast \), то
есть строк, состоящих из символов \( a, b \), при этом строка может иметь длину
\( 0 \). Будем говорить, что \emph{паттерн \( P = (P_1 P_2 \ldots P_3) \)
встречается в строке \( S = (S_1 S_2 \ldots S_n) \)}, если существует такой индекс \( i \), что
\begin{equation}
    S_i = P_1, \, 
    S_{i+1} = P_2, \,
    \ldots, \,
    S_{i+k-1} = P_k \enspace .
\end{equation}
Если паттерн \( P \) встречается в строке \( S \), и из всех таких индексов \( i
\) выбрать минимальный, то число \( (i+k-1) \) называется \emph{временем
ожидания паттерна \( P \)}.
\begin{enumerate}
\item Сколько существует бинарных строк длины \( n \), в которых не встречается
паттерн \( abb \)? 
\item Чему равно среднее время ожидания паттерна \( abb \)?
\end{enumerate}
\end{problem}       

\begin{solution}

\end{solution}

\subparagraph*{Источник задач}
\begin{enumerate}
\item Придумал из головы (скорее всего фольклор).
\item Increasing Diamonds. Olivier Bodini, Matthieu Dien, Xavier Fontaine,
Antoine Genitrini, Hsien-Kuei Hwang. LATIN-2016.
\item Expected number of crossings of the diagonal of a lattice path? 

\url{https://mathoverflow.net/a/266993/90511}
\item Philippe Flajolet, Robert Sedgewick. Analytic Combinatorics.
\end{enumerate}

\end{document}

