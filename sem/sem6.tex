\documentclass[a5paper]{article}
% Uncomment the following line to allow the usage of graphics (.png, .jpg)
%\usepackage[pdftex]{graphicx}
% Comment the following line to NOT allow the usage of umlauts

\pagestyle{empty}
\usepackage[T2A]{fontenc}
\usepackage[utf8]{inputenc}
\usepackage[russian]{babel}
\usepackage{cmap}
\usepackage{amsthm}
\usepackage{amsmath}
\usepackage{units}
\usepackage{fancyhdr}
\usepackage{forloop}
\usepackage{amssymb}
\usepackage{url}
\usepackage{hyperref}
\usepackage{xcolor}
\usepackage[inline]{enumitem}
\usepackage{graphicx}
\usepackage{caption}
\usepackage{subcaption}
\usepackage{amscd}
\usepackage[ruled,vlined]{algorithm2e}

%------------- TIKZ --------------------
\definecolor{bblue}{rgb}{0.2, 0.4, 0.8}
\definecolor{bgreen}{rgb}{0.2, 0.6, 0.4}
\definecolor{bred}{rgb}{0.8, 0.4, 0.2}
\definecolor{bviolet}{rgb}{0.7, 0.2, 0.7}
\definecolor{blackred}{rgb}{0.6, 0.3, 0.3}
\definecolor{blackblue}{rgb}{0.3, 0.3, 0.6}

\usepackage{tikz}
\usetikzlibrary{arrows,trees,shapes,snakes,shapes.geometric}
\tikzset{
  treenode/.style = {align=center, inner sep=0pt, text centered,
    font=\sffamily},
  arn_nn/.style = {treenode, circle, bblue, draw=bblue, 
    fill=bblue!10,
    minimum width=0.5em, minimum height=0.5em
},% arbre rouge noir, noeud rouge
  arn_n/.style = {treenode, circle, bblue, draw=bblue, 
    text width=1.5em, very thick,
    fill=bblue!10},% arbre rouge noir, noeud rouge
  arn_g/.style = {treenode, circle, bgreen, draw=bgreen, 
    text width=1.5em, very thick,
    fill=bblue!10},% arbre rouge noir, noeud rouge
  arn_r/.style = {treenode, circle, bviolet, draw=bviolet, 
    text width=1.5em, very thick,
    fill=bviolet!10},% arbre rouge noir, noeud rouge
  arn_x/.style = {treenode, triangle, draw=black,
    minimum width=0.5em, minimum height=0.5em},% arbre rouge noir, nil
  triangle/.style = {treenode, bred, draw=bred, fill=bred!20, regular polygon, regular polygon
    sides=3, very thick, text width=1.5em }
}

\renewcommand{\thesection}{\arabic{section}}

\renewcommand{\headrulewidth}{0.4pt}
\renewcommand{\footrulewidth}{0.4pt}

\fancyhead[R]{\thepage}
%For multipage documents only!
%\fancyfoot[L]{page: \thepage}
\fancyhead[L]{Перечислительная комбинаторика}
\fancyfoot[C]{\topic}
\pagestyle{fancy}

\renewcommand{\baselinestretch}{1.0}
\renewcommand\normalsize{\sloppypar}

\setlength{\topmargin}{-0.5in}
\setlength{\textheight}{6.5in}
\setlength{\oddsidemargin}{-0.3in}
\setlength{\evensidemargin}{-0.3in}
\setlength{\textwidth}{4.5in}
\setlength{\parindent}{0ex}

\newcounter{problemset}
\newcounter{totalpages}
%Here you should set the total number of pages
\setcounter{totalpages}{1}

\def \topic {Семинар 6}

\def \Z {\mathbb Z}
\def \R {\mathbb R}
\def \P {\mathbb P}
\def \C {\mathbb C}
\def \vec {\boldsymbol}

\theoremstyle{definition}
\newtheorem{lemma}{Лемма}
\newtheorem{example}{Пример}
\newtheorem{corollary}{Следствие}
\newtheorem*{theorem}{Теорема}
\newtheorem*{definition}{Определение}

\usepackage{titlesec}

\makeatletter
\renewcommand{\section}{\@startsection
{section}%                   % the name
{1}%                         % the level
{\z@}%                       % the indent / 0mm
{-\baselineskip}%            % the before skip / -3.5ex \@plus -1ex \@minus 
%%-.2ex
{0.5\baselineskip}%          % the after skip / 2.3ex \@plus .2ex
{\centering\large\scshape}} % the style

\renewcommand{\subsection}{\@startsection
{subsection}%                % the name
{1}%                         % the level
{\z@}%                       % the indent / 0mm
{-\baselineskip}%            % the before skip / -3.5ex \@plus -1ex \@minus 
%%-.2ex
{0.5\baselineskip}%          % the after skip / 2.3ex \@plus .2ex
{\centering\large\scshape}} % the style
\makeatother

\def\seq{\text{\textsc{seq}}}   
\def\cyc{\text{\textsc{cyc}}}   
\def\set{\text{\textsc{set}}}   
\def\C{\mathbb C}

\begin{document}

\begin{center}

\newcommand{\HRule}{\rule{\linewidth}{0.5mm}}
\HRule \\[0.2cm]
{ \Large \bfseries \topic} %\\[0.2cm]
\HRule

\end{center}

\textsc{Ключевые слова: анализ сингулярностей,
базовые приёмы асимптотического анализа коэффициентов,
гамма-функция}

\section{Введение}

Настало время рассказать, как в современном мире выглядит типичная оценка
``размера'' класса. Обычно действуют согласно следующему алгоритму.

\begin{enumerate}
\item \textbf{Комбинаторная спецификация.} Необходимо для начала сформулировать,
что мы хотим посчитать, а затем переформулировать вопрос в виде
\textit{комбинаторной спецификации}, что в принципе то же самое, что
рекуррентное уравнение. Для того, чтобы это сделать, нужно понимать внутреннюю
алгебраическую структуру объекта, знать, какие операции можно выполнять над
объектами класса (для строк это конкатенации, для деревьев это подвешивание
поддеревьев к корню, для эллиптических кривых это может быть сложение точек на
кривой, и так далее). 
\item \textbf{Функциональное уравнение.} Ваша комбинаторная спецификация, или
рекуррентное уравнение переходят в функциональное уравнение для производящей
функции.
Решения этого функционального уравнения
могут различаться по своей сложности. Например, для чисел Фибоначчи (и для любой
линейной рекурренты) решением является дробно-рациональная функция типа \(
\dfrac{1}{1 - x - x^2} \). Для чисел Каталана мы имеем дело с квадратичной
сингулярностью \( \dfrac{1 - \sqrt{1 - 4x}}{2} \), и это же верно для тернарных
деревьев (сюрприз! сингулярность это не корень кубический, а всё тот же квадратный
для любого типа деревьев). Если в уравнение входят производные, то это
дифференциальное уравнение (пример: пилообразные перестановки). Производные
обычно возникают, если нужно пометить какой-нибудь объект (скажем, наибольший),
и если конструкция рекурсивная, то получаются какие-нибудь монотонные структуры.
\item \textbf{Анализ сингулярностей.} Примеры, которые я перечислил выше,
относятся к случаям, когда уравнение можно решить \textit{явно}. Обычно это не
так. Но часто решение удаётся аппроксимировать некоторой функцией, поведение
которой мы уже знаем. Тогда это аппроксимация \textit{главного члена
асимптотики}. В принципе с помощью компьютера можно вычислять сколько угодно
членов асимптотического разложения (а если их взять достаточно много, то можно
даже надеяться, что мы получим целое число). В принципе можно считать, что
функциональное уравнение задано для функции \( f \colon \mathbb C \to \mathbb C
\), и ``детский'' случай это когда сингулярности являются полюсами, то есть
имеют как комплекснозначные функции локальную асимптотику \( \dfrac{1}{(1-z/\rho)^m}
\). На эту тему мы тоже разбирали пример: первый семинар, бюллетени и оценка
числа бюллетеней: производящая функция имела вид \( f(z) = \dfrac{1}{2 - e^z}
\), и в точке главной синглулярности её асимптотическое поведение было
эквивалентно \( -\dfrac 12 \cdot \dfrac{1}{z - \log 2} \), и число бюллетеней
равно \( \sim n! \left[ \dfrac12 \left( \dfrac{1}{\log 2} \right)^{n+1} +
O(6^{-n})\right] \).
Чем больше сингулярностей вы знаете, тем точнее асимптотическая оценка
коэффициента.
\end{enumerate}
Если вы захотите побольше узнать про эту тему, рекомендую почитать книгу
\cite{ac}. Я здесь просто перевожу фрагменты этой книги на русский язык.
\section{Примеры и принципы}
\begin{example}[Деревья Каталана и беспорядки]
Напомню конструкцию для деревьев Каталана: это корневые непомеченные деревья, то
есть любое дерево состоит из вершины и последовательности дереьвев (возможно
пустой):
\begin{equation}
    T = \bullet \times \seq(T)
    , \enspace 
    T(z) = \dfrac12 \left(
        1 - \sqrt{1 - 4z}
    \right)
    \enspace .
\end{equation}
Как известно, количество деревьев Каталана выражается числами имени этого
человека, то есть числами Каталана, а именно
\begin{equation}
    T_n = \dfrac{1}{n} {2n - 2 \choose n - 1}
    \enspace .
\end{equation}
Применим формулу Стирлинга \( n! \sim \sqrt{2 \pi n} \left( \dfrac{n}{e}
\right)^n \), получим
\begin{equation}
    T_n \sim \dfrac{4^{n-1}}{\sqrt{\pi n^3}} \enspace .
\end{equation}
Задачу о числе беспорядков тоже можно рассмотреть с точки зрения производящих
функций. Беспорядок~--- это перестановка без циклов длины 1, то есть
\begin{equation}
    D = \set(\cyc_{>1}(\bullet))
    , \enspace
    D(z) = \dfrac{e^{-z}}{1 - z}
    \enspace . 
\end{equation}
Отсюда можно явно найти коэффициент при \(z^n \):
\begin{equation}
    D_n = [z^n] D(z) = \dfrac{1}{0!} - \dfrac{1}{1!} + \ldots +
\dfrac{(-1)^n}{n!} \sim e^{-1} \enspace .
\end{equation}
Идея подхода Флажоле состоит в том, что можно получать такие оценки (и полные
асимптотические разложения) без \textit{нахождения явной формы} коэффициентов,
имея только спецификацию.

Можно заметить, что экспоненциальный рост в первом примере имеет порядок \( 4^n
\) а во втором \( 1^n \), то есть 
\begin{equation}
    \log T_n \sim n \log 4 + o(n), \quad
    \log D_n \sim n \log 1 + o(n) \enspace ,
\end{equation}
иногда для этого используют специальное обознчение \( \bowtie \):
\begin{equation}
    T_n \bowtie 4^n , \quad
    D_n \bowtie 1^n \enspace .
\end{equation}
\textbf{Первый принцип асимптотики коэффициентов.} Положение сингулярности
комплекснозначной функции \( f(z) \) диктует асимптотический рост её
коэффициентов.

\textbf{Второй принцип асимптотики коэффициентов.} Локальное поведение
комплекснозначной функции \( f(z) \) в окрестности её сингулярности определяет
субэкспоненциальный сомножитель в асимптотике.
\end{example}
В связи с этим сформулирую теорему (на самом деле она уже была сформулирована в
семинаре 1, но тогда мы очень бегло с ней ознакомились.
\begin{theorem}[{\cite[Theorem IV.10, page 258]{ac}}]
	Пусть \( f(z) \colon \C \to \C \)~--- функция, \textit{мероморфная} во всех 
	точках диска \( |z| \leq R \) (мероморфная означает, что все особенности 
	имеют тип полюсов), и имеет полюс в точке \( z = \alpha \), \( 0 < |\alpha| 
	< R 
	\). Тогда существует многочлен \( P(n) \), такой, что
	\[
		f_n = [z^n] f(z) = P(n) \alpha^{-n} + O(R^{-n}) \enspace .
	\]
    Степень многочлена \( P(n) \) равна порядку полюса минус один.
	Теорему можно обобщить на случай \( m \) полюсов, лежащих на одинаковом
расстоянии от нуля.
\end{theorem}
\begin{proof}
    В книге \cite{ac} для этой теоремы даётся два доказательства, мы рассмотрим
первое из них, хотя оба доказательства довольно сильно помогают понять более
сложные методы, которые используются в аналитической комбинаторике.

Если полюс в точке \( \alpha \) имеет порядок \( M \), то можно рассмотреть
локальное разложение
\begin{equation}
    f(z) = \sum_{k \geq -M} c_{k} (z - \alpha)^k =  S(z) + H(z) \enspace ,
\end{equation}
где \( H(z) \)~--- аналитическая функция, а \( S(z) \) имеет вид
\begin{equation}
    \dfrac{N(z)}{(z - \alpha)^M} = \dfrac{ a_0 + \ldots + a_{M-1}z^{M-1} }{(z -
\alpha)^M}
    = \sum_{k = -M}^{-1} c_k (z - \alpha)^k
    \enspace .
\end{equation}
Можно разбить коэффициент на два слагаемых
\begin{equation}
    [z^n] f(z) = [z^n] S(z) + [z^n] H(z)
\end{equation}
Для первого слагаемого можно воспользоваться формулой
\begin{equation}
    [z^n] \dfrac{1}{(z - \alpha)^r} = \dfrac{(-1)^r}{\alpha^r}
    {n+r-1 \choose r-1} \alpha^{-n}
    \enspace .
\end{equation}
Для второго слагаемого воспользуемся теоремой Коши и оценим интеграл на
окружности:
\begin{equation}
    \left|
        [z^n] H(z)
    \right|
    = \dfrac{1}{2\pi}
    \left|
        \int_{|z| = R}
        H(z) \dfrac{dz}{z^{n+1}}
    \right|
    \leq
    \dfrac{1}{2\pi}
    \dfrac{O(1)}{R^{n+1}} 2 \pi R \enspace .
\end{equation}
Здесь \( O(1) \) это \( \sup H(z) \) на окружности \( |z| = R \).
Теорема доказана.
\end{proof}
\textit{На самом деле}, если сингулярность является не полюсом, а какой-нибудь
дробной степенью типа \( \sqrt{1 - z} \), то действует похожий принцип:
\begin{equation}
    [z^n] (1 - z)^{-\alpha} \sim
    \dfrac{n^{\alpha - 1}}{\Gamma(\alpha)} 
    \left(
        1 + \dfrac{\alpha(\alpha - 1)}{2n} + \ldots
    \right)
    \enspace ,
\end{equation}
где \( \Gamma(\alpha) \) это обобщение факториала,
\begin{equation}
    \Gamma(\alpha) = \int_{0}^\infty e^{-t} t^{\alpha - 1} dt
    , \quad
    \Gamma(n+1) = n! \enspace .
\end{equation}
Я здесь не привожу доказательства этого факта, так как он потянет за собой
много интересных, безусловно, свойств Гамма-функции, но достаточно
многочисленных.

Тем не менее, теперь нельзя поступить так же, как и в предыдущей теореме, взяв
достаточно большую окрестность, которая содержит сингулярность типа квадратного
корня, потому что функция не будет мероморфной в круге. Фактически, как только
радиус окружности будет больше, чем сингулярность, функция становится
\textit{двулистной}. Надо что-то с этим делать.
\begin{theorem}[ {\cite[Theorem VI.3, page 390]{ac}}]
    Пусть заданы числа \( \phi, R \), \( R > 1 \), \( 0 < \phi < \tfrac{\pi}{2}
\). Рассмотрим открытую область
\begin{equation}
    \Delta(\phi, R) = \{
        z \colon
        |z| < R, \ 
        z \neq 1, \ 
        | \arg(z-1) | > \phi
    \}
    \enspace .
\end{equation}
Для произвольных действительных чисел \( \alpha, \beta \) рассмотрим \( \Delta
\)-аналитическую функцию
\begin{equation}
    f(z) = O \left(
         (1-z)^{-\alpha} \left(
            \log \dfrac{1}{1 -z}
         \right)^\beta
    \right)
    \enspace .
\end{equation}
Тогда
\begin{equation}
    [z^n]f(z) = O( n^{\alpha - 1} (\log n)^\beta )
    \enspace .
\end{equation}
\end{theorem}
Доказательство этой теоремы состоит из двух частей: оценки коэффициента в таком
степенной-логарифмической функции и оценка ошибки. Вторая часть требует
аккуратной оценки интеграла на каждой из частей контура. Контур из себя
представляет сложное животное, ещё сложнее, чем Пакман. Он состоит из большой
окружности, радиуса больше 1, кусочков отрезочка, которые соединяют точку \( 1
\) с большой окружностью, и маленькой окружности, с центром в точке \( 1 \) и
радиусом \( 1/n \).
\begin{example}[Асимптотика 2-регулярных графов]
Есил в помеченном графе степень каждой вершины равна \( 2 \), то он
представляет из себя множество невырожденных циклов, то есть
\begin{equation}
    R = \set(\cyc_{\geq 3}) \enspace ,
\end{equation}
где под циклом имеется в виду неориентированный цикл (число всех возможностей
нужно будет разделить на 2). Производящая функция имеет вид
\begin{equation}
    R(z) = \exp\left(
        \dfrac12 \left[
            \log \dfrac{1}{1 - z} - z - \dfrac{z^2}{2}
        \right]
    \right)
    = 
    \dfrac{e^{-z/2 - z^2/4}}{\sqrt{1-z}}
    \enspace .
\end{equation}
Представим эту функцию в виде полного асимптотического разложения по степеням \(
\sqrt{1 - z} \):
\begin{equation}
    e^{-z/2-z^2/4} = e^{-3/4} + e^{-3/4}(1-z) + \dfrac{e^{-3/4}}{4}(1-z)^2 -
\dfrac{e^{-3/4}}{12} (1-z)^3 + \ldots
\end{equation}
\begin{equation}
    R(z) \sim \dfrac{e^{-3/4}}{\sqrt{1-z}} + e^{-3/4} \sqrt{1 - z}
    + \dfrac{e^{-3/4}}{4} (1 - z)^{3/2} + \ldots
\end{equation}
Возьмём первые несколько членов и воспользуемся уточнённой формулой
для получения коэффициентов (я тоже не буду объяснять, откуда я её взял, на то
есть уточнённые формулы типа полного асимптотического разложения Стирлинга и
прочие табличные формулы):
 \begin{eqnarray*}
     [z^n] \dfrac{1}{\sqrt{1-z}}
     &\sim& \dfrac{1}{\sqrt{\pi n}} \left[
         1 - \dfrac{1}{8n} + \dfrac{1}{128n^2} + \ldots
     \right] \enspace ,
\\
     {}[z^n] \sqrt{1-z}
     &\sim& \dfrac{-1}{2\sqrt{\pi n^3}} \left[
         1 + \dfrac{3}{8n} + \ldots
     \right] \enspace ,
\\
     O ( (1-z)^{3/2} )
     &\sim& O \left(
         \dfrac{1}{n^{5/2}}
     \right) 
     \enspace ,
 \end{eqnarray*}       
откуда, складывая коэффициенты, получаем
\begin{equation}
    c_n = \dfrac{e^{-3/4}}{\sqrt{\pi n}} - \dfrac{5 e^{-3/4}}{8 \sqrt{\pi n^3}}
    + O \left(
        \dfrac{1}{n^{5/2}}
    \right)
    \enspace .
\end{equation}
\end{example}


% \section{Комплексные функции}
% Из курса ТФКП вы знаете, что комплекснозначные функции похожи на голограммы:
% если функция \textit{аналитическая}, то знание функции в сколько угодно малой
% области, по существу, определяет её единственным образом: если \( \Omega_1 \),
% \( \Omega_2 \)~--- две области, в которых функция \( f(z) \) имеет аналитическое
% продолжение, и \( \Omega_{12} \subset \Omega_1 \), \( \Omega_{12} \subset
% \Omega_2 \), то аналитические продолжения в \( \Omega_{12} \) совпадают.
% 
% Из этого наблюдения следует, что полное асимптотическое разложение можно, в
% принципе, получать двумя способами: либо \textit{очень хорошо} изучить поведение
% функции в окрестности её сингулярности, либо рассмотреть \textit{все
% сингулярности} (но как бы не очень точно). 

\bibliographystyle{plain}
\bibliography{biblio}
    
\end{document}
