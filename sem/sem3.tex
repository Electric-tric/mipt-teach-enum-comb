\documentclass{article}

% Uncomment the following line to allow the usage of graphics (.png, .jpg)
%\usepackage[pdftex]{graphicx}
% Comment the following line to NOT allow the usage of umlauts

\pagestyle{empty}
\usepackage[T2A]{fontenc}
\usepackage[utf8]{inputenc}
\usepackage[russian]{babel}
\usepackage{cmap}
\usepackage{amsthm}
\usepackage{amsmath}
\usepackage{units}
\usepackage{fancyhdr}
\usepackage{forloop}
\usepackage{amssymb}
\usepackage{url}
\usepackage{hyperref}
\usepackage{xcolor}
\usepackage[inline]{enumitem}
\usepackage{graphicx}
\usepackage{caption}
\usepackage{subcaption}
\usepackage{amscd}

\renewcommand{\thesection}{\arabic{section}}

\renewcommand{\headrulewidth}{0.4pt}
\renewcommand{\footrulewidth}{0.4pt}

\fancyfoot[L]{стр. \thepage}
\fancyfoot[R]{\today}

\fancyhead[R]{Современные приложения ДА и ФА}
%For multipage documents only!
%\fancyfoot[L]{page: \thepage}
%Uncomment this for 1-page sheets
\fancyhead[L]{Перечислительная комбинаторика}
\fancyfoot[C]{}

\pagestyle{fancy}

\renewcommand{\baselinestretch}{1.0}
\renewcommand\normalsize{\sloppypar}

\setlength{\topmargin}{-0.5in}
\setlength{\textheight}{9.1in}
\setlength{\oddsidemargin}{-0.3in}
\setlength{\evensidemargin}{-0.3in}
\setlength{\textwidth}{7in}
\setlength{\parindent}{0ex}
\setlength{\parskip}{1ex}

\newcounter{problemset}
\newcounter{totalpages}
%Here you should set the total number of pages
\setcounter{totalpages}{1}

\def \topic {Семинар 3}


\def \Z {\mathbb Z}
\def \R {\mathbb R}
\def \P {\mathbb P}
\def \C {\mathbb C}
\def \vec {\boldsymbol}
\def \seq {\text{\textsc{seq}}}

\theoremstyle{definition}
\newtheorem{lemma}{Лемма}
\newtheorem{example}{Пример}
\newtheorem*{theorem}{Теорема}
\newtheorem*{definition}{Определение}

\begin{document}

\begin{center}

\newcommand{\HRule}{\rule{\linewidth}{0.5mm}}
\HRule \\[0.2cm]
{ \Large \bfseries \topic} %\\[0.2cm]
\HRule

\end{center}

\textsc{Ключевые слова: структуры с весом, производящие функции нескольких 
переменных, цикловой индекс, теорема о композиции, алгебра производящих функций}

\textbf{Напутствие.} Производящие функции интересны как \textit{инструмент} для 
решения задач, поэтому на начальных этапах мне очень хотелось бы вам предлагать 
чисто комбинаторные задачи, для которых бы требовалось найти решение, и чтобы 
производящие функции возникали в этих решениях сами по себе, чтобы они стали 
вашим удобным инструментом. Кроме получения явного ответа (вспомните формулу 
Бине для чисел Фибоначчи) иногда полезно получить какую-нибудь рекуррентную 
формулу.

В некоторых задачах я <<запрещал>> чисто комбинаторные решения, хотя в 
педагогических целях лучше условиться так: комбинаторное решение можно 
рассказывать в дополнение к решению через производящие функции, получая за это 
дополнительные очки. У меня нет цели вас подколоть и дать задачу, где бы вы 
пытались применить этот метод напрасно: у каждой задачи есть определённая цель.

Однако я хотел бы напомнить, что из общей схемы мы 
не всегда можем \textit{явно} получить вид коэффициентов производящей функции, 
иногда мы 
даже не можем явно выразить производящую функцию, а только лишь написать 
уравнение, которому она удовлетворяет. Инструменты для работы с уравнениями мы 
будем приобретать постепенно, и поверьте, они у нас будут, наша главная цель 
это асимптотика коэффициентов, причём весьма точная. Поэтому часто будут 
возникать задачи, в которых условие звучит как <<найдите производящую 
функцию>>. При этом в воздухе начинает висеть невысказанный вопрос <<а что же 
дальше?>>. Призываю набраться терпения и отложить этот вопрос на потом.

\section{Разминка}

В этом разделе будет предложено несколько задач разной степени сложности, 
потратьте около 10 минут на попытки решить каждую из них, затем посмотрите 
решение.


\textbf{Задача.} Найдите экспоненциальную производящую функцию для количества 
инволюций, то есть таких отображений \( f \colon \{ 1, 2, \ldots, n \} \to \{ 
1, 2, \ldots, n \} \), что \( f(f(x)) = x \). Затем воспользуйтесь формулой 
свёртки для экспоненциальных производящих функций и найдите формулу для 
количества инволюций (в виде суммы).
  
\textbf{Решение.} Заметим, что инволюция является взаимно однозначным 
отображением, то есть перестановкой, причём такой, что её цикловое разложение 
содержит только циклы длины \( 1 \) или \( 2 \). Значит, такая перестановка 
состоит из \textit{множества} циклов длины \( 1 \) и \textit{множества} циклов 
длины 2, и экспоненциальная производящая функция имеет вид
\[
	I(z) = \exp \left(
		z + z^2/2
	\right) \enspace .
\]
Формула свёртки для ЭПФ:
\[
	\left(a_0 + \dfrac{a_1}{1!}x + \dfrac{a_2}{2!}x^2 + \ldots\right)
	\left(b_0 + \dfrac{b_1}{1!}x + \dfrac{b_2}{2!}x^2 + \ldots\right)	
= \sum_{n \geq 0} x^n \sum_{k = 0}^{n}{n \choose k} a_{k} b_{n-k} \enspace ,
\]
откуда число инволюций \( I_n \) равно
\[
	I_n = \sum_{k=0}^{\lfloor n/2 \rfloor}\dfrac{n!}{(n-2k)!2^k k!} \enspace .
\]

В связи с этой задачей возникает одна интересная рекуррентность, которую я решил включить в список задач на подумать.

\textbf{Задача.} Найдите экспоненциальные производящие функции для количества 
перестановок,
\begin{itemize}
	\item имеющих только циклы чётного размера,
	\item имеющих только циклы нечётного размера,
	\item имеющих чётное количество циклов,
	\item имеющих нечётное количество циклов.
\end{itemize}

Заметим, что экспоненциальная производящая функция для всевозможных перестановок
без дополнительных ограничений, имеет вид
\begin{equation}
    e^x = \sum_{k=0}^{\infty} \dfrac{n!}{n!}x^n =
\text{\textsc{set}}(\text{\textsc{cyc}})(x)
\end{equation}

Чтобы решить задачу, нужно, соответственно, найти функции
\textsc{set}$_{2k, k \in \mathbb Z_{\geq 0}}$, \textsc{cyc}$_{2k, k \in \mathbb Z_{\geq 0}}$,
\textsc{set}$_{2k + 1, k \in \mathbb Z_{\geq 0}}$, \textsc{cyc}$_{2k + 1, k \in \mathbb
Z_{\geq 0}}$.

Упражнение: докажите, что нечётная и чётная часть функции \( f(x) \) имеет вид
\[
    f_{1}(x) = \dfrac{f(x) - f(-x)}{2}, \quad
    f_{2}(x) = \dfrac{f(x) + f(-x)}{2} \enspace . 
\]

По мотивам этого упражнения будет задача на сумму цешек по модулю 3. Подумайте,
как можно обобщить этот результат на случай, когда нужна сумма коэффициентов с
индексами, которые делятся на три. Подсказка: используйте комплексные корни из
единицы третьей степени.

\textbf{Ответ.}

\begin{itemize}
	\item \( E(z) = \exp\left(\dfrac12 \log \dfrac{1}{1-z^2}\right) = 
	\dfrac{1}{\sqrt{1 - z^2}} \) , 
	\( \qquad \bullet \) \( O(z) = \exp\left(\dfrac12 \log 
	\dfrac{1+z}{1-z}\right) = 
	\sqrt{\dfrac{1+z}{1-z}} \)
	\item \( E^{*}(z) = \mathrm{ch}\ \left(\log \dfrac{1}{1 - z}\right) = 
	\dfrac12 \dfrac{1}{1 - z} + \dfrac{1 - z}{2} \) , \( \qquad \bullet \)
	\( O^{*}(z) = \mathrm{sh}\ \left( \log \dfrac{1}{1 - z}\right) = 
	\dfrac12\dfrac{1}{1 - z} + \dfrac{z - 1}{2} \).
\end{itemize}

\section{Цикловой индекс}

Здесь я следую изложению из книги Theory of Species and Tree-like Structures 
\cite{species}. Напомню, зачем вообще нужен цикловой индекс. Когда помеченные
объекты рассматриваются с точностью до изоморфизма (непомеченные объекты), то
работать с таким материалом бывает сложно, то есть сложнее, чем с помеченными
объектами. Значит, нужно подумать <<вне коробки>>, и ввести более общий объект,
который содержит в себе информацию как о помеченных, так и о непомеченных
объектах. Этот объект (формальный степенной ряд от бесконечного числа
переменных) будет более сложно устроен, но из него мы получим методы для работы
с непомеченными объектами. 

\begin{example}
	Пусть \( U \) это конечное множество вида \( \{ 1, 2, \ldots, n \} \), \( 
	\sigma \)~--- перестановка элементов \( U \). Цикловой тип перестановки --- 
	это последовательность чисел \( (\sigma_1, \sigma_2, \ldots, \sigma_n) \), 
	где \( \sigma_k \) равно количеству циклов длины \( k \). Обозначим 
	множество неподвижных точек перестановки через \( \mathrm{Fix}\; \sigma \).
	
	Пусть \( F \)~--- это класс объектов, а \( F[U] \)~--- это множество 
	всевозможных объектов, построенных на множестве атомов \( U \) (например,
    если \( U \) состоит из \( 3 \) элементов, то \( F[U] \)~--- это все трёхатомные
    объекты).
	
	Рассмотрим класс инволюций \( \mathrm{Inv} \), то есть таких функций \( 
	\psi \) из множества \( \{ 1, 2, \ldots, n \} \) в себя, что \( 
	\psi(\psi(k)) = k \), и положим \( n = 5 \). Рассмотрим перестановку \( 
	\sigma = (12)(345) \). Это не инволюция, а просто перестановка.
    Тогда каждая из 26 возможных инволюций под действием 
	этой перестановки переходит в какую-то другую инволюцию, то есть получается 
	перестановка \( \mathrm{Inv}[\sigma] \) 26-ти элементов. Эта перестановка 
	имеет цикловое разложение \( (2,0,2,0,0,3) \).

    TODO: picture.
\end{example}

\begin{definition}
	\textit{Цикловым индексом} называется формальный степеной ряд (зависящий от 
	бесконечного набора переменных \( x_1, x_2, \ldots \)) вида
	\[
		Z_F(x_1, x_2, \ldots) = \sum_{n \geq 0} \dfrac{1}{n!} \left(
			\sum_{\sigma \in S_n} |\mathrm{Fix}\ F[\sigma]|x_1^{\sigma_1} 
			x_2^{\sigma_2} \ldots
		\right) \enspace ,
	\]
	где \( (\sigma_1, \sigma_2, \ldots) \)~--- цикловой тип перестановки \( 
	\sigma \).
\end{definition}

Посмотрим внимательно на это определение. Пример с 26 инволюциями символизировал
ровно одно слагаемое из двойной суммы, а именно, если рассмотреть \( n = 5 \),
\( \sigma = (12)(345) \), то количество неподвижных точек (в нашем случае 2) и
будет коэффициентом перед мономом \( x_1^2 x_3^2 x_6^3 \). Для того, чтобы
прочувствовать определение (мы сможем начать развивать свои чувства, доказав
пару теорем и решив парочку упражнений), вполне достаточно не выписывать все 26
инволюций и изучать действие перестановки \( \sigma \),  а просто найти
количество тех инволюций, которые остаются неподвижными. Но при этом нам не надо
расслабляться: знание полного циклового разложения может оказаться полезным,
если мы захотим применить какие-нибудь новые операторы к двум известным цикловым
индексам.

\begin{example}
	Пусть \( L \)~--- класс последовательностей, \( P \)~--- класс 
	перестановок, \( S \)~--- класс множеств. Тогда
	\begin{itemize}
		\item	\( Z_L = \dfrac{1}{1 - x_1} \)
		\item 	\( Z_P = \dfrac{1}{(1-x_1)(1-x_2)\ldots} \)
		\item   \( Z_S = \exp \left(
			x_1 + \dfrac{x_2}{2} + \dfrac{x_3}{3} + \ldots
		\right) \)
	\end{itemize}
\end{example}

Доказательство этого примера достаётся вам в качестве задачи. В тексте задачи
находится наводящая идея про количество автоморфизмов перестановки.
Докажем, что цикловой индекс является обобщением обыкновенных и 
экспоненциальных производящих функций.

\begin{theorem}
Пусть \( F(x) \)~--- это экспоненциальная производящая функция для помеченных
объектов класса \( F \), \( \widetilde F(x) \)~--- обыкновенная производящая
функция для непомеченных объектов, \( Z_F(x_1, x_2, \ldots) \)~--- цикловой
индекс. Тогда выполнено:
\begin{eqnarray}
    F(x) &=&  Z_F(x, 0, 0, \ldots) \enspace , \\ 
    \widetilde F(x) &=& Z_F(x, x^2, x^3, \ldots) \enspace . 
\end{eqnarray}
\end{theorem}

\begin{proof}
    Подставим формально правую часть в определение циклового индекса. В первом случае получим:
	\[
		Z_F(x, 0, 0, \ldots) = \sum_{n \geq 0} \dfrac{1}{n!} \left(
			\sum_{\sigma \in S_n} |\mathrm{Fix}\ F[\sigma]|x^{\sigma_1} 
			0^{\sigma_2} \ldots
		\right) \enspace .
	\]
    Слагаемые в правой части будут ненулевым только при условии, что все циклы имеют длину 1.
Перестановка с таким свойством существует ровно одна~--- тождественная, и для
неё \( |\mathrm{Fix}\ F[\sigma]| \) равно количеству объектов c \( n \) атомами. 
Таким образом,  
	\[
		Z_F(x, 0, 0, \ldots) = \sum_{n \geq 0} \dfrac{1}{n!} \left(
			 |\mathrm{Fix}\ F[\sigma]| x^n
            \right) = F(x) \enspace .
	\]
Вторая часть теоремы немного сложнее, но поддаётся доказательству, если
вспомнить немного второго семестра и теории групп, а именно лемму Бёрнсайда.

Как вы помните, лемма Бёрнсайда помогала ответить на такие вопросы, например,
как найти число ожерелий, при условии, что ожерелья, которые можно совместить
поворотом или переворотом, считались эквивалентными. То есть на объектах
действует некая группа, и нас интересует число классов эквивалентности. У
каждого элемента группы \( g \in G \) можно посчитать число элементов множества,
которые он оставляет на месте, это число обозначалось \( |X^g| \), где множество
\( X \) и было тем множеством, на котором действует группа.

Если число классов эквивалентности равно \( \omega \), то 
\[
	\omega = \dfrac{1}{|G|} \sum_{g \in G} |X^g| \enspace .
\]
Значит, подставляя в определение \( \widetilde F(x) \), получим
\[
	Z_F(x, x^2, \ldots) = \sum_{n \geq 0} \left(\dfrac{1}{n!} 
		\sum_{\sigma \in S_n} |\mathrm{Fix} \; F[\sigma]|
	\right) x^n \enspace ,
\]
то есть согласно лемме Бёрнсайда, коэффициент при \( x^n \) равен числу непомеченных объектов, или числу классов эквивалентности при действии группы перестановок \( S_n  \) на объекты класса \( F \).
\end{proof}

% Обозначение \( \mathrm{aut}(\mathbf n) \), и его связь с цикловым индексом.

\subsection{Свойства циклового индекса}

Три известных нам оператора \( +, \times, \seq \) ведут себя на цикловых индексах так же, как и на уже известных нам производящих функциях:
\begin{theorem}
	\begin{eqnarray*}
		Z_{F + G}(x_1, x_2, \ldots) &=& Z_{F}(x_1, x_2, \ldots) + Z_{FG}(x_1, x_2, \ldots) \enspace , \\
		Z_{F \times G}(x_1, x_2, \ldots) &=& Z_{F}(x_1, x_2, \ldots) \cdot Z_{FG}(x_1, x_2, \ldots) \enspace , \\
		Z_{\seq(F)}(x_1, x_2, \ldots) &=& \dfrac{1}{1 - Z_{F}(x_1, x_2, \ldots)} \enspace .
	\end{eqnarray*}
\end{theorem}
\begin{proof}
	Упражнение.
\end{proof}

Свойства циклового индекса: \( +, *, seq \).

Композиция: сложные свойства.

***

Новые операции: дифференцирование, функториальная композиция, произведение 
Адамара.

\subsection{Дифференцирование и пометка}

Определение, как действует на цикловом индексе, примеры.

Оператор пометки.

Доказательство формулы Кэли для числа деревьев.

1) Дважды поинтинг некорневого дерева -- это \textit{vertebrates}. \( 
\mathfrak a^{\bullet \bullet} = V \)

2) Vertebrates -- \( V = seq_{> 0}(A) \)

3) Seq и Permutation (непустые) это эквипотентные классы, поэтому можно 
заменить, \( V = 
perm(A) \)

4) Непустые Перестановки корневых деревьев соответствуют эндофункциям.

5) Производящая функция для числа эндофункций равна \( \sum_{k > 0} n^n 
\dfrac{x^n}{n!} \).

6) Значит, \( a_n = n^{n-2} \), что и требовалось.

\subsection{Произведение Адамара}

\begin{example}
	Циклы, снабжённые подмножествами. 
	
	Производящая функция имеет вид \( \log \left( \dfrac{1}{1 - 2x} \right) \)
\end{example}

\begin{theorem}
	Произведение Адамара двух рациональных функций является рациональной 
	функцией.
\end{theorem}

\begin{proof} (Ландо)
	ПФ рациональна огда и только тогда, когда существуют числа \( q_j \) и 
	многочлены \( p_j(n) \), что
	\[
		a_n = \sum p_j(n)q_j^n \enspace .
	\]
	Произведение двух квазимногочленов является тоже квазимногочленом.
\end{proof}

\begin{example}
	Произведение Адамара \( (1 - rx)^{-1} \circ_{H} (1 - sx)^{-1}\)
\end{example}

\begin{theorem}
	Произведение Адамара на ОПФ, ЭФП, ЦИ устроена следующим образом:
\end{theorem}

\subsection{Функториальная композиция}

Найти обозначение для фунториальной композиции (маленький квадратик)

\begin{example}
	Простые графы: \( \mathfrak p {}^\square \mathfrak p^{[2]}  \)
\end{example}

Общий вид для циклового индекса: требует формулы обращения Мёбиуса.

\subsection{Полезные изоморфизмы}

\[
	(F \circ_{H} G) \square H = (F \square H) \circ_{H} (G \square H)
\]

\section{Задачи}

\begin{enumerate}
	\item(1 очко) На семинаре Эдуард предложил рекуррентное 
	соотношение для числа инволюций вида
	\[
		I_{n} = I_{n-1} + (n-1)I_{n-2} \enspace ,
	\]
	мотивируя это тем, 
	что 
	максимальный элемент можно либо зациклить с самим собой (оставшиеся \( n-1 
	\) 
	элементов образуют инволюцию), либо с каким-то из других \( (n-1) \) 
	элементов, 
	при этом остаётся \( (n-2) \) элемента, которые опять таки образуют 
	инволюцию.
	
	Проверьте, выполнено ли это соотношение, и если да, напишите 
	соответствующее 
	соотношение для экспоненциальных производящих функций, а затем глядя на 
	него, 
	придумайте (рекуррентное) соотношение для класса инволюций \( \mathrm{Inv} 
	\).
    \item(2 очка) Найдите сумму 
    \[
        \sum_{k = 0}^{\infty} {n \choose 3k}
    \]

	\item Докажите формулы для цикловых индексов \( Z_L, Z_P, Z_S \).
	\item Докажите формулу для функториальной композиции циклового индекса.
	\item \textbf{Алгебраические свойства.}
	\begin{enumerate}
	\item \( \Theta(F + G) = \Theta F + \Theta G \)
	\item	Докажите, что класс pointed sets \( \Theta E \) является 
	нейтральным элементов для операции функториальной композиции
	\[
		F \square \Theta E = \Theta E \square F = F \enspace .
	\]
	\item Произведение Адамара можно выразить с помощью декартова произведения 
	и оператора \( \square \).
	\item Свойства произведения Адамара
	\end{enumerate}
	\item Докажите, что класс ориентированных графов можно задать как 
	\[
		D = \mathfrak p \square (\Theta E \bullet \Theta E)
	\]
	\item Section 2.1 -- ex. 2. 
	\item Доказательство формулы композиции, разбить на части.
	\item Оператор \textsc{cyc}.
	\item Глянуть задачек в AC.
\end{enumerate}

\footnotesize
\bibliographystyle{plain}
\bibliography{biblio}
    
\end{document}
