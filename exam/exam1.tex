\documentclass[a5paper]{article}
% Uncomment the following line to allow the usage of graphics (.png, .jpg)
%\usepackage[pdftex]{graphicx}
% Comment the following line to NOT allow the usage of umlauts

\pagestyle{empty}
\usepackage[T2A]{fontenc}
\usepackage[utf8]{inputenc}
\usepackage[russian]{babel}
\usepackage{cmap}
\usepackage{amsthm}
\usepackage{amsmath}
\usepackage{units}
\usepackage{fancyhdr}
\usepackage{forloop}
\usepackage{amssymb}
\usepackage{url}
\usepackage{hyperref}
\usepackage{xcolor}
\usepackage[inline]{enumitem}
\usepackage{graphicx}
\usepackage{caption}
\usepackage{subcaption}
\usepackage{amscd}
\usepackage[ruled,vlined]{algorithm2e}

%------------- TIKZ --------------------
\definecolor{bblue}{rgb}{0.2, 0.4, 0.8}
\definecolor{bgreen}{rgb}{0.2, 0.6, 0.4}
\definecolor{bred}{rgb}{0.8, 0.4, 0.2}
\definecolor{bviolet}{rgb}{0.7, 0.2, 0.7}
\definecolor{blackred}{rgb}{0.6, 0.3, 0.3}
\definecolor{blackblue}{rgb}{0.3, 0.3, 0.6}

\usepackage{tikz}
\usetikzlibrary{fit,arrows,trees,shapes,snakes,shapes.geometric,calc}
\tikzset{
  treenode/.style = {align=center, inner sep=0pt, text centered,
    font=\sffamily},
  arn_nn/.style = {treenode, circle, bblue, draw=bblue, 
    fill=bblue!10,
    minimum width=0.5em, minimum height=0.5em
},% arbre rouge noir, noeud rouge
  arn_n/.style = {treenode, circle, bblue, draw=bblue, 
    text width=1.5em, very thick,
    fill=bblue!10},% arbre rouge noir, noeud rouge
  arn_g/.style = {treenode, circle, bgreen, draw=bgreen, 
    text width=1.5em, very thick,
    fill=bblue!10},% arbre rouge noir, noeud rouge
  arn_r/.style = {treenode, circle, bred, draw=bred, 
    text width=1.5em, very thick,
    fill=bred!10},% arbre rouge noir, noeud rouge
  arn_x/.style = {treenode, triangle, draw=black,
    minimum width=0.5em, minimum height=0.5em},% arbre rouge noir, nil
  triangle/.style = {treenode, bred, draw=bred, fill=bred!20, regular polygon, regular polygon
    sides=3, very thick, text width=1.5em }
}

\renewcommand{\thesection}{\arabic{section}}

\renewcommand{\headrulewidth}{0.4pt}
\renewcommand{\footrulewidth}{0.4pt}

\fancyhead[R]{\thepage}
%For multipage documents only!
%\fancyfoot[L]{page: \thepage}
\fancyhead[L]{Перечислительная комбинаторика}
\fancyfoot[C]{\topic}

\renewcommand{\baselinestretch}{1.0}
\renewcommand\normalsize{\sloppypar}

\setlength{\topmargin}{-1.2in}
\setlength{\textheight}{6.8in}
\setlength{\oddsidemargin}{-0.3in}
\setlength{\evensidemargin}{-0.3in}
\setlength{\textwidth}{4.5in}
\setlength{\parindent}{0ex}

\newcounter{problemset}
\newcounter{totalpages}
%Here you should set the total number of pages
\setcounter{totalpages}{1}

\def \topic {Экзамен по курсу перечислительной комбинаторики 30.04.2017}

\def \Z {\mathbb Z}
\def \R {\mathbb R}
\def \P {\mathbb P}
\def \C {\mathbb C}
\def \vec {\boldsymbol}

\theoremstyle{definition}
\newtheorem{lemma}{Лемма}
\newtheorem{example}{Пример}
\newtheorem{corollary}{Следствие}
\newtheorem{problem}{Задача}
\newtheorem*{theorem}{Теорема}
\newtheorem*{definition}{Определение}

\usepackage{titlesec}

\makeatletter
\renewcommand{\section}{\@startsection
{section}%                   % the name
{1}%                         % the level
{\z@}%                       % the indent / 0mm
{-\baselineskip}%            % the before skip / -3.5ex \@plus -1ex \@minus 
%%-.2ex
{0.5\baselineskip}%          % the after skip / 2.3ex \@plus .2ex
{\centering\large\scshape}} % the style
 
\renewcommand{\thesection}{\thechapter\arabic{section}.}

\renewcommand{\subsection}{\@startsection
{subsection}%                % the name
{1}%                         % the level
{\z@}%                       % the indent / 0mm
{-\baselineskip}%            % the before skip / -3.5ex \@plus -1ex \@minus 
%%-.2ex
{0.5\baselineskip}%          % the after skip / 2.3ex \@plus .2ex
{\centering\large\scshape}} % the style
\makeatother

\def\seq{\text{\textsc{seq}}}   
\def\cyc{\text{\textsc{cyc}}}   
\def\set{\text{\textsc{set}}}   
\def\C{\mathbb C}
\newcommand{\HRule}{\rule{\linewidth}{0.5mm}}

\begin{document}
\begin{center}

\vspace{-2cm}
\HRule \\[0.2cm]
{ \Large \bfseries \topic } %\\[0.2cm]
\HRule

\end{center}
\begin{problem}
    Рассмотрим правило для преобразования подстрок
\begin{eqnarray*}
    BBA &\to& AAB \enspace , \\
    CBA &\to& CAB \enspace , \\
    CCA &\to& CCB \enspace . 
\end{eqnarray*}
Докажите, что минимальное число шагов, которое требуется для перехода от строки
\( CC A^{n} \) к строке, заканчивающейся на \( B \), имеет
асимптотику \( \widetilde\Theta(\phi^n) \), \( \phi = \frac{1 + \sqrt{5}}{2} \),
где \( \widetilde \Theta \) означает ``с точностью до полиномиального по \( n \)
множителя''.

При желании можете найти точное необходимое количество шагов.
\end{problem}

\begin{problem}
Пусть на окружности равномерно расставлены точки с номерами
\( 1, 2, \ldots, n \), и точка с номером \( 1 \) находится на северном полюсе
окружности.
Обозначим \( T_n \) количество деревьев, построенных на этих точках, таких, что
хорды окружности, соответствующие рёбрам дерева, не пересекаются.
\begin{enumerate}
\item[(i)] Докажите, что производящая функция количества этих деревьев
\begin{equation}
    T(z) = T_1 z + T_2 z^2 + T_3z^3 + \ldots
\end{equation}
удовлетворяет уравнению
\(
    T^3(z) - z T(z) + z^2 = 0
\).
\item[(ii)] Пользуясь формулой обращения Лагранжа, покажите, что
\begin{equation}
    T_n = \dfrac{1}{2n - 1} {3n - 3 \choose n-1}
    \enspace .
\end{equation}
\end{enumerate}
\end{problem}

\begin{problem}
    Серёжа любит доминошки \( 1 \times 1 \) и \( 1 \times 2 \) и складывает
из них полоски \( 1 \times n \). Назовём их доминошками первого типа и
доминошками второго типа, соответственно.
 Каждый раз когда в последовательности доминошек
следующая по счёту доминошка имеет другой тип, Серёжа выпивает стакан чая.
Например, если последовательность типов имеет вид \( 11122121 \), то Серёжа
выпьет \( 4 \) стакана чая.
Найдите математическое ожидание (можно асимптотически в зависимости от \( n \))
количество стаканов чая, которое выпьет Серёжа, если ему попалась полоска длины
\( n \), выбранная случайно равновероятно из множества всех таких полосок. 
\end{problem}

\begin{problem}
    Рассмотрим множество \emph{плоских корневых деревьев}, содержащих \( n \)
вершин. Под плоским корневым деревом понимается дерево, у которого выделена одна
вершина (корневая) и порядок потомков вершины имеет значение, то есть граф
нарисован на плоскости.

Найдите (можно асимптотически) математическое ожидание количества таких вершин, у
которых ровно один потомок.
\end{problem}

\begin{problem}
\emph{Возрастающее тернарное дерево}~--- это дерево, построенное на вершинах с
различными метками от \( 1 \) до \( n \), такое, что что у каждой вершины
либо \( 0 \) либо \( 3 \) потомка, и номера меток вдоль каждого пути возрастают.
Докажите, что количество возрастающих тернарных деревьев на \( n \) вершинах 
равно \( n! 2^{-n} { 2n \choose n} \).
\end{problem}

\vspace{1cm}

\begin{center}
\begin{tabular}{|p{10cm}|}
\hline
\textbf{Правила письменного экзамена: }
\begin{enumerate}
\item Можно решать любые три задачи или больше. Засчитываются три задачи с
максимальным результатом плюс карма за идеи по остальным задачам.
\item Время сдачи задания 1 мая 2017 года, время 21:00 мск.
\item Можно пользоваться любой литературой, в том числе рекомендуемой
литературой по курсе. Можно искать в интернете конкретные теоремы и
факты, но запрещается искать в интернете условия задач в явном виде (вряд ли это
приведёт к успеху, просто это не совсем этично).
\item Задачи нужно оформить и прислать в отчётливом читаемом виде по адресу 
\texttt{vit.north@gmail.com} или личным сообщением в \texttt{Telegram} по номеру
\texttt{+79154140726}. Можно прислать
фотографии, но с хорошей резкостью, правильным поворотом и хорошим разрешением.
Можно оформлять в \LaTeX, если время позволяет.
\end{enumerate}
\\
\hline
\end{tabular}
\end{center}

\end{document}

