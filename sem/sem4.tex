\documentclass{article}

% Uncomment the following line to allow the usage of graphics (.png, .jpg)
%\usepackage[pdftex]{graphicx}
% Comment the following line to NOT allow the usage of umlauts

\pagestyle{empty}
\usepackage[T2A]{fontenc}
\usepackage[utf8]{inputenc}
\usepackage[russian]{babel}
\usepackage{cmap}
\usepackage{amsthm}
\usepackage{amsmath}
\usepackage{units}
\usepackage{fancyhdr}
\usepackage{forloop}
\usepackage{amssymb}
\usepackage{url}
\usepackage{hyperref}
\usepackage{xcolor}
\usepackage[inline]{enumitem}
\usepackage{graphicx}
\usepackage{epstopdf}
\usepackage{caption}
\usepackage{subcaption}
\usepackage{amscd}

\renewcommand{\thesection}{\arabic{section}}

\renewcommand{\headrulewidth}{0.4pt}
\renewcommand{\footrulewidth}{0.4pt}

\fancyfoot[L]{стр. \thepage}
\fancyfoot[R]{\today}

\fancyhead[R]{Современные приложения ДА и ФА}
%For multipage documents only!
%\fancyfoot[L]{page: \thepage}
%Uncomment this for 1-page sheets
\fancyhead[L]{Перечислительная комбинаторика}
\fancyfoot[C]{}

\pagestyle{fancy}

\renewcommand{\baselinestretch}{1.0}
\renewcommand\normalsize{\sloppypar}

\setlength{\topmargin}{-0.5in}
\setlength{\textheight}{9.1in}
\setlength{\oddsidemargin}{-0.3in}
\setlength{\evensidemargin}{-0.3in}
\setlength{\textwidth}{7in}
\setlength{\parindent}{0ex}
\setlength{\parskip}{1ex}

\newcounter{problemset}
\newcounter{totalpages}
%Here you should set the total number of pages
\setcounter{totalpages}{1}

\def \topic {Семинар 3}


\def \Z {\mathbb Z}
\def \R {\mathbb R}
\def \P {\mathbb P}
\def \C {\mathbb C}
\def \vec {\boldsymbol}
\def \seq {\text{\textsc{seq}}}
\def \fprod {\; \raisebox{.3\height}{\scalebox{0.6}{$\square$}}\; }
\def \point {\raisebox{.1\height}{\scalebox{0.8}{$\Theta$}}}

\theoremstyle{definition}
\newtheorem{lemma}{Лемма}
\newtheorem{example}{Пример}
\newtheorem*{theorem}{Теорема}
\newtheorem*{definition}{Определение}

\usepackage{titlesec}

\makeatletter
\renewcommand{\section}{\@startsection
{section}%                   % the name
{1}%                         % the level
{\z@}%                       % the indent / 0mm
{-\baselineskip}%            % the before skip / -3.5ex \@plus -1ex \@minus 
%%-.2ex
{0.5\baselineskip}%          % the after skip / 2.3ex \@plus .2ex
{\centering\large\scshape}} % the style

\renewcommand{\subsection}{\@startsection
{subsection}%                % the name
{1}%                         % the level
{\z@}%                       % the indent / 0mm
{-\baselineskip}%            % the before skip / -3.5ex \@plus -1ex \@minus 
%%-.2ex
{0.5\baselineskip}%          % the after skip / 2.3ex \@plus .2ex
{\centering\large\scshape}} % the style
\makeatother

\begin{document}

\begin{center}

\newcommand{\HRule}{\rule{\linewidth}{0.5mm}}
\HRule \\[0.2cm]
{ \Large \bfseries \topic} %\\[0.2cm]
\HRule

\end{center}

\textsc{Ключевые слова: 
структуры с весом, доказательство теоремы о композиции для структур с весом,
полиномы Эрмита, формула Харера-Цагира
}

\section{Разминка}

\textbf{Вопрос}. Что означает последовательность букв \( F[U] \)?

\textbf{Ответ}. Из контекста символов и нашего курса должно быть
восстановлено следующее: \( F \) это некоторый класс объектов, \( U \)~--- это
множество атомов вида \( U = \{ 1, 2, \ldots, n \} \), \( F[U] \)~--- это
<<срезка>> класса \( F \) для количества атомов \( n \), иначе говоря, множество
всевозможных объектов размера \( n \) данного класса. Или множество всех
объектов, построенных на множестве атомов \( U \).

\textbf{Вопрос}. Что означает набор букв \( F[\sigma] \)?

\textbf{Ответ}. Из контекста следует, что \( \sigma \) это скорее всего
перестановка \( n \) элементов, то есть \( \sigma \in S_n \). \( F[\sigma]\)
означает <<транспорт>> вдоль перестановки \( \sigma \). Пусть множество \( F[n]
\) содержит \( \ell \) объектов. Тогда \( F[\sigma] \) задаёт некоторую
перестановку этих \( \ell \) объектов.

\textbf{Вопрос.} Что же такое тогда класс объектов?

\textbf{Ответ.} Зафиксируем некоторое число \( n \in \mathbb N \), и рассмотрим
множество \( U = \{ 1, 2, \ldots, n \} \). Класс объектов порождает множество \(
F[U] \) объектов размера \( n \) со следующим дополнительным свойством. На
множестве перестановок \( \sigma \in
S_n \) задано отображение \( \sigma \to F[\sigma] \), которое обладает
свойствами
\begin{enumerate}
 \item[a)] \( F[\sigma \circ \tau] = F[\sigma]\circ F[\tau] \).
 \item[b)] \( F[Id_n] = Id_{F[U]} \). 
\end{enumerate}

\textbf{Басня о декатегорификации стада овец (\cite{category_feynman},
\cite[Remark 6, page 11]{species}).}
Давным-давно, когда пастухи хотели понять, \textit{изоморфны} ли два стада овец,
они искали конкретный изоморфизм. Для этого было необходимо выстроить в ряд оба
стада, а затем сопоставить каждой овце из первого стада овцу из второго стада.
В один прекрасный день, пастух изобрёл декатегорификацию. Пастух понял, что овец
можно <<сосчитать>>, установив изоморфизм между множеством овец и множеством
натуральных чисел, используя бессмысленные слова типа <<один, два, три>>.
Сравнивая числа, можно было понять, изоморфны ли два стада, не предъявляя
соответсвующий изоморфизм! Короче говоря, множество \( \mathbb N \) было создано
как декатегорификация FinSet, категория конечных множеств, чьи морфизмы это
отображения конечных множеств.

Затем пастухи изобрели основные операции типа сложения, умножения,
декатегорифицировав важные теоретико-множественные операции: непересекающееся
объединение, декартово произведение, и так далее. Затем, их потомки расширили
понятие чисел, изобрели ещё более замечательные формальные операции и их
свойства: изобрели рациональные, действительные, комплексные числа, функции,
интегралы и производные, надоказывали теорем. В процессе этой деятельности
оригинальная связь с категорией конечных множеств была утеряна.

Производящая функция это выражение типа \( \sum_{n \geq 0} \dfrac{a_n}{n!} x^n
\), \( a_n \in \mathbb N \), и над такими выражениями можно проделывать разные
формальные операции. Категория \textit{классов объектов} это
категорифицированная версия кольца формальных степенных рядов. Более того,
свойства a) и b) позволяют сформулировать определение \textit{класса объектов}
на языке теории категорий: класс объектов \( F \) это \textit{функтор}
\[
    F \colon \mathbb B \to \mathbb E
\] 
из категории \( \mathbb B \) конечных множеств в категорию \( \mathbb E \)
конечных множеств \textit{и функций}.


% \section{Формула обращения Мёбиуса}
% \begin{theorem}
%     Пусть задана последовательность \( a_n \in \mathbb C \), через которую
% определена другая последовательность
% \[
%     S_n = \sum_{d | n} a_d \enspace .
% \]
% Тогда последовательность \( a_n \) можно выразить через \( S_n \) с помощью
% формулы обращения Мёбиуса:
% \[
%     a_n = \sum_{d | n} \mu (n / d) S_d \enspace ,
% \]
% где \( \mu(n) \)~--- функция Мёбиуса, заданная формулой
% \[
%     \mu(n) = \begin{cases}
%         (-1)^k, & n = p_1 p_2 p_3 \ldots p_k \enspace , \\
%         1, & n = 1 \enspace , \\
%         0, & p^2 \mid n \enspace .
% \end{cases}
% \]
% \end{theorem}
% \begin{proof}
%     Рассмотрим дзета-функцию Римана
% \[
%     \zeta(s) = \sum_{k \geq 1} \dfrac{1}{k^s} = \prod_{p \in \mathcal P} \left(
% 1 - \dfrac{1}{p^s} \right)^{-1} \enspace ,
% \]
% где \( \mathcal P \)~--- множество простых чисел. Последнее соотношение следует
% из того, что \( \dfrac{1}{1 - x} = \sum_{k \geq 0} x^k \), а любое число
% имеет каноническое разложение на простые сомножители. (Из того, что \( \zeta(1)
% = \infty \), в частности, следует, как заметил Эйлер, что простых чисел
% бесконечно много.)
% 
% Если обратить дзета-функцию, получим функцию Мёбиуса
% \[
%     M(s) = \sum_{n \geq 1} \dfrac{\mu(n)}{n^s} \enspace .
% \]
% Теперь заметим, что если последовательности \( a_n \), \( S_n \) удовлетворяют
% соотношению \( S_n = \sum_{d | n} a_d \), то можно выписать соотношение для
% \textit{производящих функций Дирихле}:
% \[
% \left(
%    \sum_{k \geq 1} \dfrac{1}{k^s} \cdot \sum_{k \geq 1} \dfrac{a_k}{k^s} = \sum_{k \geq 1}
% \dfrac{S_k}{k^s} 
% \right)
% \quad \Leftrightarrow \quad
% \left(
%     \sum_{k \geq 1} \dfrac{a_k}{k^s} = \zeta(s) \cdot \sum_{k \geq 1}
% \dfrac{S_k}{k^s} 
% \right)
% \]
% (Для того, чтобы проветить этот факт, необходимо получить формулу свёртки для
% производящих функций Дирихле). Следовательно, обращая \( \zeta(s) \), получаем:
% \[
% \left(
%     \sum_{k \geq 1} \dfrac{S_k}{k^s} = M(s) \sum_{k \geq 1} \dfrac{a_k}{k^s}
% \right)
% \quad \Leftrightarrow \quad
% \left(
%     a_n = \sum_{d | n} \mu (n / d) S_d 
% \right) \enspace .
% \]
% \end{proof}
% \begin{example}
%     Функция \( F(n) \) задана на всём натуральном ряду. Любую ли такую функцию
% \( F(n) \) можно представить в виде
% \[
%     F(n) = \sum_{d | n} f(d) \enspace ?
% \]
% \end{example}
% \begin{example}
%     Как доказать, что \( n = \sum_{d | n} \varphi(d) \)?
% \end{example}
\section{Структуры с весом}
В задачах перечислительной комбинаторики часто необходимо рассматривать разные
параметры изучаемых структур. Например, при анализе алгоритмов, хороших и
разных, нужно, например, посчитать число деревьев с заданным количичеством
вершин \( n \) и количеством листовых вершин \( k \), или даже с заданной
высотой \( h \).

    Оказывается, понятие взвешенной структуры интересно не только само по себе,
но и необходимо для доказательства теоремы о композиции.

\begin{example}
    Рассмотрим класс корневых деревьев \( \mathcal A \), и каждому корневому
дереву \( \alpha \in \mathcal A \) дополнительно сопоставим \textit{вес} \(
w(\alpha) \), равный
\[
    w(\alpha) = t^{f(\alpha)} \enspace ,
\]
где \( t \)~--- формальная переменная, \( f(\alpha) \)~--- число листьев
\(\alpha\). Это позволяет сгруппировать корневые деревья в соответствии с
дополнительным параметром <<число листьев>>. Будем говорить, что множество \(
\mathcal A[U] \) является \textit{взвешенным}, а также что переменная \( t \)
выступает в роли <<счётчика>> для числа листьев.

Взвешенное количество деревьев \( |\mathcal A[U]|_w \) определяется как сумма
весов \( w(\alpha) \):
\[
    | \mathcal A[U] |_w = \sum_{\alpha \in \mathcal A[U]} w(\alpha) = \sum_{\alpha \in
\mathcal A[U]} t^{f(\alpha)} \enspace .
\]
Простая перегруппировка слагаемых показывает, что при \( |U| = n \),
\[
    | \mathcal A[U] |_w = \sum_{k=0}^n a_{n,k} t^k \enspace ,
\]
где \( a_{n,k}\) равно количеству корневых деревьях на \( n \) вершинах, имеющих
\( k \) листьев. Подстановка \( t = 1 \) даёт эффект подсчёта каждого дерева с
весом \( 1 \), и тогда \( | \mathcal A[U] | = n^{n-1} \).
\end{example}

Обратите внимание, что понятие взвешенных классов позволяет существенно
расширить множество классов, для которых их производящие функции аналитичны в
некоторой точке. Например, рассмотрим класс всевозможных графов на \( n \) вершинах,
имеющих \( k \) рёбер. Производящая функция имеет вид
\[
    f(x, t) = \sum_{n = 0}^\infty (1 + t)^{ {n \choose 2} } x^n \enspace ,
\]
Эта функция аналитична в точке \( (x, t) = (0, -1) \).
Более того, теперь можно рассматривать классы, в которых содержится бесконечное
число объектов заданного размера \( n \), главное, чтобы \( |\mathcal A[U]|_w \)
было корректно определено, то есть коэффициент при каждом мономе вида \( \vec
t^{\vec f(\alpha)} \) был конечным.

Для взвешенных объектов также определено понятие \textit{изоморфизма}, и здесь
возникает дополнительное требование, что изоморфизм должен сохранять вес.

\begin{definition}
    Для \(w\)-взвешенной структуры \( F_w \) определены экспоненциальная
производящая функция, обыкновенная производящая функция и цикловой индекс:
\[
    F_w(x) = \sum_{n \geq 0} |F[n]|_w \dfrac{x^n}{n!} \enspace ,
\]
\[
    Z_{F_w} (x_1, x_2, \ldots) = \sum_{n \geq 0} \dfrac{1}{n!} \left(
        \sum_{\sigma \in S_n} |\mathrm{Fix}\; F[\sigma]|_w x_1^{\sigma_1}
x_2^{\sigma_2} \ldots
    \right)
    \enspace ,
\]
где \( |\mathrm{Fix}\; F[\sigma]|_w \) это вес множества, состоящего из
фиксированных точек под действием перестановки \( \sigma \). Другими словами,
множество \( \mathrm{Fix}\; F[\sigma] \) осталось из предыдущей конструкции, а
вес его считается теперь не как количество элементов, а как сумма весов.

Автоморфизмы сохраняют вес, следовательно отношение экивалентности определено
корректно, и 
\[
     \widetilde F_w (x) = \sum_{n \geq 0} |F[n] / \sim |_w x^n \enspace .
\]
\end{definition}

Заметим, что формулы \( F_w(x) = Z_{F_w} (x, 0, 0, \ldots) \) и \( \widetilde
F_{w} (x) = Z_{F_w} (x, x^2, x^3, \ldots) \) остаются верными. 

\begin{example}[Формула Харера-Цагира, \cite{lando, pittel}]
\label{example:harer-zagier}
    В топологии известно утверждение, что всякая замкнутая ориентируемая
двухмерная поверхность гомеоморфна сфере, к которой приклеено конечное число
ручек. Поверхностью рода\footnote{Буква \( g \) используется в соответствии с
английским термином \textit{genus}.} \( g \) называется 
двумерная сфера с приклеенными к ней \( g \) ручками. Поверхность рода \( 0 \)
это просто сфера, поверхность рода \( 1 \) это тор. 

Замкнутые ориентируемые поверхности можно изготавливать из многоугольников,
склеивая их стороны попарно. Например, склеивание противоположных сторон
квадрата даёт тор. Рассмотрим производящую функцию от двух переменных
\[
    T(s, t) = 1 + 2 s \sum_{n = 0}^\infty \dfrac{T_n(t)}{(2n-1)!!} s^n \enspace
, 
\]
где \( T_n(t) = \sum_{g} \varepsilon_g(n) t^{n+1-2g} \)~--- производящая функция
для числа склеек \( \varepsilon_g(n) \) согласно роду поверхности при заданном
числе вершин (число \( n - 2g + 1 \) равно числу вершин, согласно формуле Эйлера
\( V-E+F = 2 - 2g \)).
Смысл двойного факториала в знаменателе объясняется следующим: число
всевозможных склеек в точности равно \( (2n-1)!! = 1 \cdot 3 \cdot \ldots \cdot
(2n-1) \), а значит, и число \( \dfrac{\varepsilon_g(n)}{(2n-1)!!} \) равно
вероятности получить склейку рода \( g \) при случайном склеивании, то есть
сумма всех весов, как и положено при \( t = 1 \), равна единице. 

Для этой производящей функции существует удивительно простое и изящное выражение 
\[
    T(s, t) = \left(
        \dfrac{1+s}{1-s}
    \right)^t \enspace .
\]
\end{example}

\begin{example}[Полиномы Эрмита, {\cite[Example 16, page 89]{species}}]
    Рассмотрм взвешенный класс инволюций \( \mathrm{Inv}_w \) (то есть
перестановок \( \varphi \) со свойством \( \varphi \circ \varphi = \mathrm{Id}
\)), таких, что вес каждой перестановки равен
\[
    w(\varphi) = t^{\varphi_1} (-1)^{\varphi_2} \enspace ,
\]
где \( \varphi_1, \varphi_2 \)~--- это, как обычно, количество фиксированных
точек и количество циклов длины \( 2 \), соответственно. Выполнено комбинаторное
равенство
\[
    \mathrm{Inv}_w = \text{\textsc{set}}(X_t + (\mathcal C_2)_{-1}) \enspace ,
\]
где \( X_t \) это класс, состоящий из одного атома с весом \( t \), а \(
\mathcal C_2 \)~--- цикл размера 2. Если принять на веру тот факт, что формулы
для композиции взвешенных классов работают <<примерно похожим образом>>, то
получается ЭПФ
\[
    \mathrm{Inv}_w(x) = \exp \left(
        tx - \dfrac12 x^2
    \right) = \sum_{n \geq 0} H_n(t) \dfrac{x^n}{n!} \enspace ,
\]
где ВНЕЗАПНО \( H_n(t) \) оказывается полиномом Эрмита от переменной \( t \)
степени \( n \). К вашему сведению, полиномы Эрмита являются собственными
функциями \textit{квантового гармонического осциллятора} (подробно об этом
рассказывают на физтехе в курсе квантовой механики в 6-7 семестрах
\cite[section 4.1]{mipt-quantum}), и выражаются
формулой
\[  
    H_n(t) = (-1)^n e^{t^2} \dfrac{d^n}{dt^n} e^{-t^2}
    \enspace .
\]
Ещё полиномы Эрмита являются решениями дифференциального уравнения
\[
    H_n(t)'' - t H_n(t)' + n H_n(t) = 0 \enspace ,
\]
что можно выяснить комбинаторно. Кроме того, можно показать, что эти полиномы
удовлетворяют рекуррентному соотношению
\[
    H_{n+1} (t) = t H_n(t) - n H_{n-1}(t) \enspace .
\]
\end{example}

\section{Теорема о композиции}
    Мы собираемся доказать формулу
\[
    \widetilde{F_w \circ G_v} (x) = Z_{F_w} (\widetilde G_v(x), \widetilde
G_{v^2}(x^2) ,
\widetilde G_{v^3}(x^3), \ldots ) \enspace ,
\]
а затем увидим, что формула для композиции циклового индекса \( Z_{F_{w} \circ
G_{v}} =
Z_{F_w} \circ Z_{G_v} \) следует из предыдущей формулы, используя тот факт, что
некоторые симметрические функции являются алгебраически независимыми.

Напомним, что для того, чтобы получить обыкновенную производящую функцию
некоторого класса объектов, мы рассматривали объекты с точностью до
автоморфизма. Для каждого числа атомов \( n \) было задано действие группы \(
S_n \) на всевозможных объектах размера \( n
\).

Пусть задан класс объектов \( F = F_w \). Сопоставим этому классу объектов новый класс
\( \widetilde F = \widetilde F_w \), чья экспоненциальная производщая функция является
обыкновенной производящей функцией для класса \( F \). Интуитивно говоря, структура
на \( \widetilde F \)-объекте дополнительно снабжена некоторым автоморфизмом \( F
\)-структуры этого объекта.
\begin{definition}
    Пусть задан класс объектов \( F \). Ему можно сопоставить класс \( \widetilde
F \), определённый следущим образом. Для любого множества атомов \( U \)
\[
    \widetilde F[U] = \Big\{
        (s, \alpha) \in F[U] \times \text{\textsc{perm}}[U] \mid
         \alpha \cdot s = s
    \Big\} \enspace ,
\]
где вес задаётся формулой \(w(s, \alpha) = w(s) \), а  транспорт вдоль перестановки задан формулой
\[
    \widetilde F[\sigma](\sigma, \alpha) = (\sigma \cdot s, \sigma \circ \alpha
\circ \sigma^{-1} ) \enspace .
\]
\end{definition}
С помощью леммы Бёрнсайда несложно показать, что ЭПФ для \( \widetilde F  \)
равна \( \widetilde F(x) \).

% ------------------

\textbf{Здесь будут промежуточные теоремы, которые устанавливают факт для
композиции ОПФ взвешенных структур}

% ------------------

\begin{theorem}
    Пусть задан класс объектов \( F = F_w \) с весом из \( \mathbb A \). Тогда цикловой индекс \( Z_{F_w} \)~--- это
единственный формальный степенной ряд \( f(x_1, x_2, x_3, \ldots) \), такой, что
для любого класса объектов \( G \) выполнено
\[
    \widetilde{F_w \circ G_v}(x) = f(\widetilde G_v(x), \widetilde G_{v^2}(x^2), \widetilde
G_{v^3}(x^3), \ldots) \enspace .
\] 
\end{theorem}
\begin{proof}
    Мы уже выяснили, что цикловой индекс \( f = Z_{F_w} \) удовлетворяет
указанному соотношению. Докажем единственность. Выберем переменные \( t_1, t_2,
t_3, \ldots \), которые не принадлежат кольцу \( \mathbb A \), а затем
рассмотрим класс объектов
\[
    G_v = X_{t_1} + X_{t_2} + X_{t_3} + \ldots
\]
Для него выполнено
\[
    \widetilde{G_{v^k}}(x^k) = (t_1^k + t_2^k + t_3^k + \ldots) x^k = S_k x^k
\enspace ,
\]
где \( S_k \)~--- симметрическая сумма степеней \( k \).
Известный способ показать, что две функции совпадают на всевозможных наборах
аргументов~--- это рассмотреть их разность, как функцию от того же набора
аргументов, и доказать, что она всегда равна нулю. Наш путь~--- рассмотреть не
все наборы аргументов, а только те, которые мы указали выше в качестве примера с
симметричными функциями. Следовательно, для того, чтобы придерживаться такого
плана, достаточно доказать,
что для функции \( f \in \mathbb A [ [ x ] ] \) выполнено
\[
    \forall \vec t, x \ 
    f(S_1 x, S_2 x^2, S_3 x^3, \ldots ) = 0 \quad 
    \Rightarrow \quad
    \forall \vec x\ 
    f(x_1, x_2, x_3, \ldots) = 0 \enspace .
\]
Если \( f(x_1, x_2, x_3, \ldots) = \sum_{n_1, n_2, \ldots} a_{n_1,n_2, \ldots}
x_1^{n_1} x_2^{n_2} \ldots \), то положим
\[
    f_n(x_1, x_2, x_3, \ldots) = \sum_{n_1 + 2n_2 + \ldots = n}
    a_{n_1, n_2, \ldots} x_1^{n_1} x_2^{n_2} \ldots \enspace .
\] 
Тогда \( f_n \) является полиномом от \( x_i \) и 
\begin{eqnarray*}
    f(S_1 x, S_2 x^2, S_3 x^3, \ldots) = 0 & \Rightarrow & 
    \sum_{n \geq 0} f_n(S_1, S_2, S_3, \ldots) x^n = 0\\
    & \Rightarrow & f_n(S_1, S_2, S_3, \ldots) = 0 , \\
    & \Rightarrow & f_n(x_1, x_2, x_3, \ldots) = 0, \\
    & \Rightarrow & f(x_1, x_2, \ldots) = 0 \enspace .
\end{eqnarray*}
в силу того, что симметрические функции \( S_k = S_k(t_1, t_2, \ldots) \)
являются алгебраически независимыми\footnote{Мне кажется, это легко доказать с
помощью определителя Вандермонда, хоть и число переменных бесконечно. Если бы
существовала конечная линейная комбинация, это бы привело к конечномерному
противоречию.} над \( \mathbb A \). 
\end{proof}

\textbf{Здесь будет продолжение доказательства.}

\section{Задачи}

\begin{enumerate}
    \item(2 очка) Выпишите производящую функцию для числа склеек \( 2n \)-угольника,
дающих тор. Можно воспользоваться книгой Ландо \cite{lando}, там нет решения, но
есть целая глава, посвящённая этой теме.
    \item(1 очко) Докажите, что функции \( T_n(t) \) из примера
\ref{example:harer-zagier}
является чётной при нечётном \( n \) и нечётной при чётном \( n \).
\end{enumerate}

\footnotesize
\bibliographystyle{plain}
\bibliography{biblio}
    
\end{document}
