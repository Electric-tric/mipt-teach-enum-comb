\documentclass{article}
% Uncomment the following line to allow the usage of graphics (.png, .jpg)
%\usepackage[pdftex]{graphicx}
% Comment the following line to NOT allow the usage of umlauts

\pagestyle{empty}
\usepackage[T2A]{fontenc}
\usepackage[utf8]{inputenc}
\usepackage[russian]{babel}
\usepackage{cmap}
\usepackage{amsthm}
\usepackage{amsmath}
\usepackage{units}
\usepackage{fancyhdr}
\usepackage{forloop}
\usepackage{amssymb}
\usepackage{url}
\usepackage{hyperref}
\usepackage{xcolor}
\usepackage[inline]{enumitem}
\usepackage{graphicx}
\usepackage{caption}
\usepackage{subcaption}
\usepackage{amscd}


\renewcommand{\thesection}{\arabic{section}}

\renewcommand{\headrulewidth}{0.4pt}
\renewcommand{\footrulewidth}{0.4pt}

\fancyfoot[L]{стр. \thepage}
\fancyfoot[R]{(Серёжа Довгаль)}

\fancyhead[R]{Математический кружок}
%For multipage documents only!
%\fancyfoot[L]{page: \thepage}
%Uncomment this for 1-page sheets
\fancyhead[L]{Гимназия 8}
\fancyfoot[C]{}

\pagestyle{fancy}

\renewcommand{\baselinestretch}{1.0}
\renewcommand\normalsize{\sloppypar}

\setlength{\topmargin}{-0.5in}
\setlength{\textheight}{9.1in}
\setlength{\oddsidemargin}{-0.3in}
\setlength{\evensidemargin}{-0.3in}
\setlength{\textwidth}{7in}
\setlength{\parindent}{0ex}
\setlength{\parskip}{1ex}

\newcounter{problemset}
\newcounter{totalpages}
%Here you should set the total number of pages
\setcounter{totalpages}{1}


\def \topic {Элементы теории графов}

\def \Z {\mathbb Z}
\def \R {\mathbb R}
\def \P {\mathbb P}
\def \C {\mathbb C}
\def \vec {\boldsymbol}

\theoremstyle{definition}
\newtheorem{problem}{Задача}
\newtheorem{solution}{Задача}

\newtheorem{lemma}{Лемма}
\newtheorem*{example}{Пример}
\newtheorem*{theorem}{Теорема}
\newtheorem*{definition}{Определение}

\usepackage{titlesec}

\makeatletter
\renewcommand{\section}{\@startsection
{section}%                   % the name
{1}%                         % the level
{\z@}%                       % the indent / 0mm
{-\baselineskip}%            % the before skip / -3.5ex \@plus -1ex \@minus 
%%-.2ex
{0.5\baselineskip}%          % the after skip / 2.3ex \@plus .2ex
{\centering\large\scshape}} % the style
\makeatother

\begin{document}

\begin{center}

\newcommand{\HRule}{\rule{\linewidth}{0.5mm}}
\HRule \\[0.2cm]
{ \Large \bfseries \topic} %\\[0.2cm]
\HRule

\end{center}

\textsc{Ключевые слова: определение графа, дерево, пути в графе, лемма Рамсея.}

\begin{definition}
	\textit{Графом} называется конечное множество точек, некоторые из которых 
	соединены линиями. Точки называются \textit{вершинами} графа, а соединяющие 
	линии~--- \textit{рёбрами.}
\end{definition}

\begin{definition}
	\textit{Путь} в графе~--- это последовательность вершин, 
	соединённых рёбрами. \textit{Цикл}~--- это путь, в котором первая и 
	последняя вершины совпадают. \textit{Простой путь}~--- такой путь в котором 
	все вершины различны. \textit{Простой цикл}~--- такой цикл, в котором все 
	вершины, кроме первой и последней, различны.
\end{definition}

\begin{definition}
	\textit{Степень вершины} в графе~--- это количество рёбер, которые из неё 
	торчит.
\end{definition}

\section*{Разминка}

\begin{problem}
	На плоскости нарисовано \( n \) точек. Затем между всеми парами точек 
	провели соединяющие линии. Сколько получилось линий?
\end{problem}

\begin{problem}
	На плоскости нарисовано \( n \) точек. Сколько различных графов с вершинами 
	в этих точках можно нарисовать?
\end{problem}

\begin{problem}
	Можно ли на плоскости нарисовать \( 9 \) отрезков так, чтобы каждый отрезок 
	пересекался ровно с тремя другими?
\end{problem}

\begin{problem}
	На вечеринке каждый человек знаком хотя бы с десятью другими людьми. 
	Докажите, что можно составить хоровод из 11 человек, такой, что каждый 
	человек знаком со своими соседями.
\end{problem}

\section*{Задачи поинтереснее}

\begin{definition}
	\textit{Расстояние} между двумя вершинами~--- это длина кратчайшего пути 
	между этими вершинами.
\end{definition}

\begin{example}
	Рассмотрим строки длины \( n \), составленные из нулей и единиц. Если две 
	строки различаются ровно в одном символе, то такие строки можно соединить 
	ребром. \textit{Расстояние} между строками~--- это минимальное количество 
	символов, которое нужно заменить, чтобы из одной строки получить другую.
\end{example}

\begin{problem}
	Шпионы Икс и Игрек передают друг другу сообщения, являющиеся строками длины 
	\( 3 \), из нулей и единиц. При передаче может возникнуть помеха, которая 
	исказит один любой символ и заменит его на другой. Секретный шифр~--- это 
	строка из нулей и единиц длиной \( 2016 \) символов. Какое минимальное 
	количество сообщений необходимо для передачи секретного шифра?
\end{problem}

\begin{definition}
	Вершина графа называется \textit{центральной}, если наибольшее расстояние 
	до других вершин минимально. Это расстояние называется \textit{радиусом} 
	графа.
\end{definition}

\begin{example}
	Давайте пофантазируем на тему того, как можно составлять графы из различных 
	объектов, не имеющих к графам, на первый взгляд, никакого отношения. Мы уже 
	умеем составлять графы из людей и их знакомств, из систем отрезков, и даже 
	из строк, составленных из нулей и единиц.
	
	Рассмотрим строки, составленные из символов \( 0, 1, 2, \ldots, n \). У 
	строки определён \textit{циклический сдвиг}. Например, если взять строку \( 
	01120 \), то её циклическим сдвигом будет \( 11200 \), а если ещё раз 
	сдвинуть циклически, получится \( 12001 \). Я предлгаю вам посмотреть, что 
	будет, если строки рассматривать как вершины графа, и соединить строку и её 
	циклический сдвиг ребром. Это подсказка к следующей задаче.
\end{example}

\begin{problem}
	Напомню, что число \( p \) называется \textit{простым}, если у него нет 
	делителей кроме единицы и самого себя. Докажите \textit{малую теорему 
	Ферма}: если \( p \)~--- простое число, то для любого натурального \( n \) 
	число \( n^p - n \) делится на \( p 
	\).
\end{problem}

\section*{Деревья и их свойства}

\begin{definition}
	Граф называется \textit{связным}, если между любыми двумя вершинами есть 
	путь.
	
	\textit{Деревом} называется связный граф, не имеющий циклов.
\end{definition}

\begin{problem}
	Дерево тогда и только тогда, когда:
	
	1) Связный граф с \( n \) вершинами, \( n-1 \) рёбрами.
	
	2) Любые две вершины соединены единственным путём
	
\end{problem}

\begin{problem}
	В дереве есть хотя бы две висячих вершины.
\end{problem}

\begin{problem}
	Докажите, что из любого связного графа можно удалить одну вершину так, 
	чтобы он 
	остался связным.
\end{problem}

Это к решению предыдущей задачи.
\begin{problem}
	У любого связного графа есть остовное дерево.
\end{problem}


%Молекула углеводорода является связным графом, причём валентность углерода 
%равна 4, а валентность водорода равна 1. Найдите максимальное число атомов 
%водорода в молекуле, содержащей ровно \( n \) атомов углерода, для каждого \( 
%n 
%\).
%
%\textbf{Историческая справка}. Число различных структурных изомеров алканов 
%можно оценить с помощью теоремы Редфилда-Пойя~--- теоремы из перечислительной 
%комбинаторики и теории групп.

\begin{problem}
	В стране 15 городов, некоторые из которых соединены авиалиниями, 
	принадлежащими 
	трём авиакомпаниям. Известно, что даже если любая из авиакомпаний прекратит 
	полёты, можно будет добраться из любого города в любой другой (возможно с 
	пересадками), пользуясь рейсами оставшихся компаний. Какое наименьшее 
	количество авиалиний может быть в стране?
	
	В графе 15 вершин, его рёбра покрасили в 3 цвета. Известно, что если 
	выкинуть 
	рёбра любого цвета, граф останется связным. Какое наименьшее количество 
	рёбер в 
	графе?
\end{problem}


\section*{Решения}

\begin{solution}
	Первую вершину можно выбрать \( n \) способами, вторую \( n-1 \) способами 
	(потому что линия соединяет различные вершины). Получается \( n \cdot (n-1) 
	\) пар вершин. При этом каждая линия посчитана дважды. Значит, итоговое 
	количество рёбер \( \frac{n(n-1)}{2} \).
\end{solution}

\begin{solution}
	Как мы выяснили в предыдущей задаче, максимальное количество рёбер в таком 
	графе равно \( \frac{n(n-1)}{2} \). Каждое ребро можно либо провести, либо 
	не провести. Значит, количество возможностей для кажого ребра равно \( 2 
	\). Ответом является произведение двоек в количестве \( \frac{n(n-1)}{2} \) 
	штук, то есть
	\[
		\underbrace{2 \cdot 2 \cdot \ldots \cdot 2}_{n(n-1)/2} = 
		2^{\frac{n(n-1)}{2}}
	\]
\end{solution}

\begin{solution}
	\textit{Лемма о рукопожатиях}. Если в графе степень каждой вершины нечётна, 
	то число вершин в графе чётно.
	
	\begin{proof} Пусть в графе имеется \( n \) вершин.
		Пусть степени вершин равны \( d_1, d_2, \ldots, d_n \). Сложим эти 
		числа. Заметим, что полученная сумма равна числу пар вершин, 
		соединённых рёбрами. Но таким образом, каждое ребро посчитано дважды, 
		значит сумма должна делиться на \( 2 \). Так как все числа \( d_i \) 
		нечётны, то значит, число \( n \) чётно.
	\end{proof}
	Перейдём к решению задачи. Поставим в соответствие каждому отрезку вершину 
	графа. Две вершины соединим ребром, если соответствующие им отрезки 
	пересекаются. Таким образом, мы получили граф на 9 вершинах, где степень 
	каждой вершины равна 3. Противоречие с леммой о рукопожатиях.
\end{solution}

\begin{solution}
	Любой граф \( G \) содержит путь длины \( \delta(G) \) и цикл длины хотя бы 
	\(  \delta(G) + 1 \).
\end{solution}

\section*{Задачи для самостоятельного решения}

\begin{definition}
	\textit{Ориентированным графом} называется конечное множество точек, 
	некоторые из которых соединены направленными отрезками (стрелочками). 
	Другими словами, это граф, в котором каждое ребро имеет направление.
\end{definition}

\begin{problem}
	Пусть задан ориентированный граф на \( n \) вершинах, между каждыми двумя 
	вершинами которого в некотором направлении проведено ребро. Докажите, что в 
	нём можно выбрать не более одного ребра, и изменить, если нужно, его 
	направление так, чтобы в графе нашёлся цикл, проходящий по всем вершинам 
	графа.
\end{problem}

\begin{problem}
	Степень каждой вершины в графе не более 3. Докажите, что вершины можно 
	покрасить в два цвета так, чтобы у каждой вершины было не более одного 
	соседа её цвета.
	
	\textbf{Подсказка:} Сначала покрасьте вершины произвольным образом, а затем 
	пробуйте перекрашивать вершины так, чтобы состояние графа <<приближалось>> 
	к тому, которое требуется в задаче.
\end{problem}

\begin{problem}
	Докажите следующее обобщение леммы о рукопожатиях. Назовём вершину 
	\textit{плохой}, если её степень в графе нечётна. Докажите, что если в 
	графе нечётное число вершин, то число плохих вершин чётно.
\end{problem}


\footnotesize

\textbf{Литература}

О.И. Мельников --- Занимательные задачи по теории графов; Н.В. Горбачёв --- 
сборник олимпиадных задач по математике


    
\end{document}
